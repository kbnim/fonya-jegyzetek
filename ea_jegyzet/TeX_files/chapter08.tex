\chapter{Szintaktikus elemzés}

\section{Grammatikai előfeltételek}

A lexikális elemzés kinyerte a tokenek sorozatát a forrásfájlból. Ebben a lépésben az a feladatunk, hogy ezen tokenekből a ``nyelvtani hierarchiát'', a \textbf{szintaxisfát} állítsuk fel. Ehhez szükségünk vannak a \textbf{környezetfüggetlen nyelvtanok}ra, valamint az ezek elfogadására szolgáló \textbf{veremautomaták}ra.

Szintkaktikus elemzőt manapság nagyon egyszerűen hozhatunk létre különböző generátorok segítségével. Ilyen például a \href{https://www.gnu.org/software/bison/}{Bison}, amit a gyakorlaton is használunk. Ennek a forrásfájljában (pl. \texttt{while.y}) megadjuk a lehetséges tokeneket és definiáljuk a szabályainkat.

Ahhoz, hogy elemezhető nyelvet tudjunk készíteni, a nyelvtanunknak szüksége van arra, hogy bizonyos előfeltételeket teljesítsen.

\begin{enumerate}
	\item \underline{Redukáltság}: Nincsenek ``felesleges'' nemterminálisok. \\
	Mindegyik nemterminálishoz adható olyan levezetés, amiben szerepel, és nem üres terminális sorozatot vezetünk le belőle.
	
	\item \underline{Ciklusmentesség}: Nincs $\prodrule{A}{\text{\textsuperscript{+}\textit{A}}}$ levezetés. \\
	Ciklusos nyelvtan olyan, aminek az egyik bal oldalának levezetéséből visszajuthatunk önmagába. Példa ciklusos (tehát nem jó) nyelvtanra:
	\begin{flalign*}
		& \prodrule{S}{A} \\
		& \prodrule{A}{a \mid B} \\
		& \prodrule{B}{A}.
	\end{flalign*}
	
	\item \underline{Egyértelműség}: Minden szóhoz \textbf{pontosan egy szintaxisfa} tartozik.
	\begin{itemize}
		\item Több levezetés tartozhat egy szóhoz, de a szintaxisfáik legyenek identikusak!
		
		\begin{minipage}{0.5\linewidth}
			\begin{flalign*}
				S \Longrightarrow AB &\Longrightarrow  aB \Longrightarrow ab \\
				S \Longrightarrow AB &\Longrightarrow  Ab \Longrightarrow ab \\
			\end{flalign*}
		\end{minipage}
		\begin{minipage}{0.5\linewidth}
			\begin{forest}
				for tree={parent anchor=south, child anchor=north, l=1em, l sep=1em, s sep=1em, edge={-}}
				[$S$
					[$A$
						[$a$]
					]
					[$B$
						[$b$]
					]
				]
			\end{forest}
		\end{minipage}
		
		\pagebreak
		
		\item Példa nem egyértelmű nyelvtanra: $\prodrule{S}{\texttt{utasítás} \mid SS}$.
		
		\begin{minipage}{0.5\linewidth}
			\begin{forest}
				for tree={ edge={-}}
				[$S$
					[$S$
						[\texttt{utasítás}, tier=terminal]
					]
					[$S$
						[$S$
							[\texttt{utasítás}, tier=terminal]
						]
						[$S$
							[\texttt{utasítás}, tier=terminal]
						]
					]
				]
			\end{forest}
		\end{minipage}
		\begin{minipage}{0.5\linewidth}
			\begin{forest}
				for tree={ edge={-}}
				[$S$
					[$S$
					[$S$
					[\texttt{utasítás}, tier=terminal]
					]
					[$S$
					[\texttt{utasítás}, tier=terminal]
					]
					]
					[$S$
						[\texttt{utasítás}, tier=terminal]
					]
				]
			\end{forest}
		\end{minipage}
		
		Ez a nemegyértelműség feloldható a nyelvtan átalakításával:
		\[ \prodrule{S}{\texttt{utasítás} ~ S \mid \texttt{utasítás}} ~~~ \text{ vagy } ~~~ \prodrule{S}{S ~ \texttt{utasítás} \mid \texttt{utasítás}}. \]
	
		\begin{minipage}{0.5\linewidth}
			\begin{forest}
				for tree={ edge={-}}
				[$S$
				[\texttt{utasítás}, tier=terminal]
				[$S$
				[\texttt{utasítás}, tier=terminal]
				[$S$
				[\texttt{utasítás}, tier=terminal]
				]
				]
				]
			\end{forest}
		\end{minipage}
		\begin{minipage}{0.5\linewidth}
			\begin{forest}
				for tree={ edge={-}}
				[$S$
				[$S$
				[$S$
				[\texttt{utasítás}, tier=terminal]
				]
				[\texttt{utasítás}, tier=terminal]
				]
				[\texttt{utasítás}, tier=terminal]
				]
			\end{forest}
		\end{minipage}
	
		\item A nem-egyértelműség feloldható, ha megadjuk az operátorok precedenciáját és asszociativitását. Az alábbi nyelvtan nem egyértelmű:
		\[ \prodrule{E}{\texttt{szám} \mid E \texttt{ + } E \mid E \texttt{ * } E}. \]
		
		\begin{minipage}{0.5\linewidth}
			\begin{forest}
				for tree={ edge={-}}
				[$E$
					[$E$
						[\texttt{szám}, tier=terminal]
					]
					[\texttt{*}, tier=terminal]
					[$E$
						[$E$
							[\texttt{szám}, tier=terminal]
						]
							[\texttt{+}, tier=terminal]
						[$E$
							[\texttt{szám}, tier=terminal]
						]
					]
				]
			\end{forest}
		\end{minipage}
		\begin{minipage}{0.5\linewidth}
			\begin{forest}
				for tree={ edge={-}}
				[$E$
				[$E$
				[$E$
				[\texttt{szám}, tier=terminal]
				]
				[\texttt{*}, tier=terminal]
				[$E$
				[\texttt{szám}, tier=terminal]
				]
				]
				[\texttt{+}, tier=terminal]
				[$E$
				[\texttt{szám}, tier=terminal]
				]
				]
			\end{forest}
		\end{minipage}
	
		\begin{minipage}{0.5\linewidth}
			\begin{forest}
				for tree={ edge={-}}
				[$E$
				[$E$
				[\texttt{szám}, tier=terminal]
				]
				[\texttt{+}, tier=terminal]
				[$E$
				[$E$
				[\texttt{szám}, tier=terminal]
				]
				[\texttt{+}, tier=terminal]
				[$E$
				[\texttt{szám}, tier=terminal]
				]
				]
				]
			\end{forest}
		\end{minipage}
		\begin{minipage}{0.5\linewidth}
			\begin{forest}
				for tree={ edge={-}}
				[$E$
				[$E$
				[$E$
				[\texttt{szám}, tier=terminal]
				]
				[\texttt{+}, tier=terminal]
				[$E$
				[\texttt{szám}, tier=terminal]
				]
				]
				[\texttt{+}, tier=terminal]
				[$E$
				[\texttt{szám}, tier=terminal]
				]
				]
			\end{forest}
		\end{minipage}
		
		Átalakítva:
		\begin{itemize}
			\item A \texttt{*} magasabb precedenciájú, mint a \texttt{+}.
			\item Mindkét operátor balasszociatív.
		\end{itemize}
		\begin{flalign*}
			&\prodrule{E}{F \mid E ~ \texttt{+} ~ F} \\
			&\prodrule{F}{\texttt{szám} \mid F ~ \texttt{*} ~ \texttt{szám}}.
		\end{flalign*}
	
	\end{itemize}
\end{enumerate}

\section{Felülről lefele elemzés}

A \textbf{felülről lefele elemzés} az egyik lehetséges stratégiája a szintaktikus elemzésnek. A \textbf{startszimbólumból indulva a terminálisok felé} építjük a szintaxisfát. A \textbf{bemenet feldolgozása balról jobbra} történik, így ezáltal \textbf{legbaloldalibb levezetés}t állít elő -- ami azt jelenti, hogy több terminális esetén a legbaloldalibbat helyettesíti.

Szemléltessük az alábbi nyelvtanon:
\begin{flalign*}
	& \prodrule{S}{AB} \\
	& \prodrule{A}{a} \\
	& \prodrule{B}{bc}.
\end{flalign*}
Legyen a bemeneti szövegünk: $abc$. A szó szintaxisfáját így kapjuk meg:

\begin{figure}[h!]
	\centering
\begin{minipage}{0.2\linewidth}
	\begin{center}
		\begin{forest}
			for tree={ edge={draw=none}}
			[$S$
			[$ $
			[$a$, tier=terminal]
			]
			[$ $
			[$b$, tier=terminal]
			[$c$, tier=terminal]
			]
			]
		\end{forest}
		
		$S$
	\end{center}
\end{minipage}
\begin{minipage}{0.2\linewidth}
	\begin{center}
		\begin{forest}
			for tree={ edge={-}}
			[$S$
			[$A$
			[$a$, tier=terminal, edge={draw=none}]
			]
			[$B$
			[$b$, tier=terminal, edge={draw=none}]
			[$c$, tier=terminal, edge={draw=none}]
			]
			]
		\end{forest}
		
		$S \Rightarrow AB$
	\end{center}
\end{minipage}
\begin{minipage}{0.2\linewidth}
	\begin{center}
		\begin{forest}
			for tree={ edge={-}}
			[$S$
			[$A$
			[$a$, tier=terminal]
			]
			[$B$
			[$b$, tier=terminal, edge={draw=none}]
			[$c$, tier=terminal, edge={draw=none}]
			]
			]
		\end{forest}
		
		$S \Rightarrow AB \Rightarrow aB$
	\end{center}
\end{minipage}
\begin{minipage}{0.30\linewidth}
	\begin{center}
		\begin{forest}
			for tree={ edge={-}}
			[$S$
			[$A$
			[$a$, tier=terminal]
			]
			[$B$
			[$b$, tier=terminal]
			[$c$, tier=terminal]
			]
			]
		\end{forest}
		
		$S \Rightarrow AB \Rightarrow aB \Rightarrow abc$
	\end{center}
\end{minipage}
	\caption{Felülről lefele elemzés lépései}
\end{figure}

%\subsection{Szabály kiválasztása}

Felmerül a kérdés: \textbf{mi alapján választjuk ki a használandó szabályt?} A probléma egyszerűen feloldható előreolvasással. Vegyük a következő példát!

Legyen a nyelvtanunk, ami vesszővel (\texttt{v}) elválasztott elemek (\texttt{e}) listáját írja le. Megengedjük az üres listát is ($\emptyword$).
\begin{flalign*}
	& \prodrule{S}{\emptyword \mid \texttt{e}F} \\
	& \prodrule{F}{\emptyword \mid \texttt{ve}F}
\end{flalign*}
 
A példaszövegünk legyen ``\texttt{apple, banana, pear}''. A lexikális elemzővel megkapjuk a tokenek sorozatát, ami ``\texttt{eveve}''. Megelőlegezzük, hogy a szintaxisfának így kell kinéznie.

\begin{figure}[h!]
	\centering
	\begin{forest}
		for tree={edge={-}}
		[$S$
			[\texttt{e}, tier=terminal]
			[$F$
				[\texttt{v}, tier=terminal]
				[\texttt{e}, tier=terminal]
				[$F$
					[\texttt{v}, tier=terminal]
					[\texttt{e}, tier=terminal]
					[$F$
						[$\emptyword$, tier=terminal]
					]
				]
			]
		]
	\end{forest}
\end{figure}

Szemléltessük a szintaktikus elemzést! Egyelőre nem olvastunk egy karaktert sem, emellett a szintaxisfánk is kizárólag az $S$ startszimbólumból áll. Beolvasunk egyet, a szövegünk \texttt{e} lesz. Megnézzük, hogy erre melyik szabály passzol. Mivel az $\prodrule{S}{\texttt{e}F}$ jobb oldala ugyanezzel a karakterrel kezdődik, így ezt választjuk. Így a fánk már 3 csúcsból áll.

\begin{figure}[h!]
	\centering
	\begin{forest}
		for tree={edge={-}}
		[$S$
		[\texttt{e}, tier=terminal]
		[$F$, tier=terminal]
		]
	\end{forest}
\end{figure}

Folytatjuk az elemzést, előreolvasunk ismét egy karaktert. A szövegünk állapota \texttt{ev}, hisz \texttt{v}-t olvastunk be. Ezt kihasználva kiválasztjuk az $\prodrule{F}{\texttt{ve}F}$ szabályt. A szintaxisfánk állapota:

\begin{figure}[h!]
	\centering
	\begin{forest}
		for tree={edge={-}}
		[$S$
			[\texttt{e}, tier=terminal]
			[$F$
				[\texttt{v}, tier=terminal]
				[\texttt{e}, tier=terminal]
				[$F$, tier=terminal]
			]
		]
	\end{forest}
\end{figure}

Ezt az eljárást addig folytatjuk, ameddig fel nem dolgoztuk a teljes szöveget. Ha nem marad már beolvasandó karakter, akkor az $\emptyword$-t kapjuk, ami biztosítja, hogy befejezhessük az elemzést. Ha felidézzük a végleges fát, a legvégén láthattunk egy elsőre feleslegesnek tűnő üres szót. Valójában pont emiatt került a végére.

%\subsection{Előreolvasás mennyisége}

Következő kérdés: \textbf{hány tokent kell előreolvasnunk?}  Szerencsére, ezt is megválaszolhatjuk, ugyanis ez a nyelvtan tulajdonságaitól függ. Az előző nyelvtan esetében elegendő volt 1-et előre olvasnunk, míg más nyelvtanok esetében más lehet ez a konstans.

Ennek jellemzéséhez bevezetjük az \framebox{$LL(k)$ nyelvtan} fogalmát.

\begin{tcolorbox}
	\begin{definition}[$LL(k)$ nyelvtan]
		Egy környezetfüggetlen nyelvtan $LL(k)$ tulajdonságú valamely $k \in \mathbb{N}^+$ számra,
		ha felülről lefelé elemzés esetén a legbaloldalibb feldolgozatlan
		nemterminálishoz egyértelműen meghatározható a rá alkalmazandó
		nyelvtani szabály legfeljebb k token előreolvasásával.
	\end{definition}
\end{tcolorbox}

Az elnevezés a ``\textit{\textbf{l}eft to right using \textbf{l}eftmost derivation}'' elnevezés angol rövidítéséből származik. A legutóbbi példánk $LL(1)$ tulajdonságú.

\textbf{\textit{Megjegyzések}}.
\begin{itemize}
	\item Azt mondtuk, hogy a megfelelő szabály kiválasztása előreolvasással oldható meg. Ez azt feltételezi, hogy nulla karaktert nem olvashatunk előre, ezért $k \in \mathbb{N}^+$.
	\item Nem minden nyelvtanhoz adható meg ilyen konstans. A következő tétel ezt mondja ki (nem bizonyítjuk).
\end{itemize}

\begin{tcolorbox}
	\begin{theorem}
		Van olyan nyelvtan, ami semmilyen
		$k \in \mathbb{N}^+$-re sem $LL(k)$.
	\end{theorem}
\end{tcolorbox}

Például az alábbi nyelvtanhoz nem létezik megfelelő $k \in \mathbb{N}^+$ szám.
\begin{flalign*}
	&\prodrule{S}{A | B} \\
	&\prodrule{A}{\texttt{a} | \texttt{a}A} \\
	&\prodrule{B}{\texttt{ab} | \texttt{a}B\texttt{b}}
\end{flalign*}

%\subsection{Rekurzív leszállás}

A tanulmányaink során az $LL(1)$ nyelvtanokkal foglalkozunk részletesebben. Ennek egy implementációja az ún. \framebox{\textbf{rekurzív leszállás}}.

A rekurzív leszállást azért kedveljük, mivel rendkívül kényelmessé teszi az elemző lekódolását. Gyakorlatilag arra van szükségünk, hogy a \textbf{nyelvtani szabályokat} közvetlenül \textbf{átírjuk függvényekké} egy tetszőleges programozási nyelvben. 
\begin{enumerate}
	\item \textbf{Mindegyik nemterminálishoz írunk egy-egy függvényt}. \\
	A példák a korábban definiált ``listás nyelv'' nemterminálisait illusztrálják.
	
	\item \textbf{Minden szabályalternatívát egy-egy elágazás ágaként fejezünk ki}. \\
	Például a $\prodrule{S}{\emptyword \mid \texttt{e}F}$ szabály két ágból fog állni; egy az üres szó esetéért felel, a másik meg a lista fejeleméért. \\
	Gondoskodnunk kell a hibakezelésről is, emiatt egy további ágat fentartunk erre a célra. Így végső soron egy 3-ágú elágazásunk lesz.
	
	\item \textbf{Az ágak belsejében a szabály jobboldalának minden szimbólumához egy-egy utasítást rendelünk}. \\
	A terminálisokhoz egy-egy $accept()$ függvényhívás fog tartozni, míg a nemterminálisokhoz a hozzájuk tartozó eljárás kerül meghívásra(az $S$-hez az $S()$, az $F$-hez az $F()$).
	
	\item Az $accept()$ eljárás feladatai:
	\begin{itemize}
		\item Ha az elvárt token következik, akkor új tokent kér a lexikális elemzőtől.
		\item Egyébként hibát jelez.
	\end{itemize}
\end{enumerate}


\begin{figure}[h!]
	\centering
	
	~\\
	
\begin{stuki*}{/* $\prodrule{S}{\emptyword \mid \texttt{e}F}$ */ ~ S()}
	\begin{CASE}{3}{3}
		\WHEN{\stm{next = \texttt{end}}}
		\stm*[2]{/* $\prodrule{S}{\emptyword}$ */ \\ SKIP}
		\WHEN{\stm{next = \texttt{e}}}
		\stm{\text{/* } \prodrule{S}{\texttt{e}F} \text{ */}}
		\stm{accept(\texttt{e})}
		\stm{F()}
		\WHEN{\stm*{default}}
		\stm{error()}
	\end{CASE}
\end{stuki*}

~\\

\begin{stuki*}{/* $\prodrule{F}{\emptyword \mid \texttt{ve}F}$ */ ~ F()}
	\begin{CASE}{4}{3}
		\WHEN{\stm{next = \texttt{end}}}
		\stm*[3]{/* $\prodrule{F}{\emptyword}$ */ \\ SKIP}
		\WHEN{\stm{next = \texttt{v}}}
		\stm{\text{/* } \prodrule{F}{\texttt{ve}F} \text{ */}}
		\stm{accept(\texttt{v})}
		\stm{accept(\texttt{e})}
		\stm{F()}
		\WHEN{\stm*{default}}
		\stm{error()}
	\end{CASE}
\end{stuki*}

~\\

\begin{stuki*}{accept$(t : Token)$}
	\begin{IF}{1}{\stm{next = \texttt{t}}}
		\stm{next := lexer.next()}
		\ELSE
		\stm{error()}
	\end{IF}
\end{stuki*}
	\caption{A példanyelvtan elemzőjének függvényeinek struktogramjai}
\end{figure}

\newpage

\textbf{\textit{Megjegyzések}}.
\begin{itemize}
	\item Ha a nyelvtan rekurzív, akkor rekurzív vagy kölcsönösen rekurzív függvényeket kapunk, innen a módszer neve.
	
	\item Valójában ez az elemző is veremautomata: a függvényhívásokat kezelő futási idejű verem az automata verme.
	
	\item A levezetés legbaloldalibb redukálható nemterminálisát \textbf{nyél}nek is nevezik.
\end{itemize}

Az egyes függvények felépítését elég alaposan körül tudjuk írni. Azonban felmerülhet a kérdés, hogy az \textbf{elágazások feltételeit miképpen tudjuk meghatározni}? A válasz a $FIRST$ halmaz és $FOLLOW$ halmaz fogalmában rejlik, amiket be is vezetünk.

\begin{tcolorbox}
	\begin{definition}[$FIRST_1$ halmaz]
		Adott nyelvtan esetén egy $\alpha$ szimbólumsorozatra a $FIRST_1 (\alpha)$ halmaz
		azokat a \textbf{terminálisokat} tartalmazza, amelyek az $\alpha$-ból levezethető
		szimbólumsorozatok elején állnak. 
		
		Ha $\alpha$-ból levezethető az üres szó
		($\emptyword$), akkor $\emptyword$ is eleme a halmaznak.
	\end{definition}
\end{tcolorbox}

A $FIRST$ halmaz általánosan $n$ hosszú eredménysorozatokra
is definiálható: $FIRST_n(\alpha)$.
\begin{flalign*}
	FIRST_1(\emptyword) & = \{\emptyword\} ~~~~~~~~~ \underline{\emptyword} \\
	FIRST_1(\texttt{e}F) & = \{ \texttt{e} \} ~~~~~~~~~ \underline{\texttt{e}}F \\
	FIRST_1(\texttt{ve}F) & = \{ \texttt{v} \} ~~~~~~~~ \underline{\texttt{v}}\texttt{e}F \\
	FIRST_1(F) & = \{ \emptyword, \texttt{v} \} ~~~~~~ \underline{\emptyword} \text{ és } \underline{\texttt{v}}\texttt{e}F
\end{flalign*}

\begin{tcolorbox}
	\begin{definition}[$FOLLOW_1$ halmaz]
		Adott nyelvtan esetén egy $\alpha$ szimbólumsorozatra a $FOLLOW_1 (\alpha)$ halmaz
		azokat a \textbf{terminálisokat} tartalmazza, amelyek az $\alpha$ után
		állhatnak a kezdőszimbólumból induló levezetésekben.
		
		Ha $\alpha$ után nem áll semmi, akkor \# (szöveg vége jel) eleme a halmaznak.
	\end{definition}
\end{tcolorbox}

A $FOLLOW$ halmaz általánosan $n$ hosszú eredménysorozatokra
is definiálható: $FOLLOW_n(\alpha)$.
\begin{flalign*}
	FOLLOW_1(S) & = \{ \# \} ~~~~~~~~~ S\underline{~~} \\
	FOLLOW_1(F) & = \{ \# \} ~~~~~~~~~ S \Rightarrow \texttt{e}F\underline{~~} \Rightarrow \texttt{ve}F\underline{~~} \Rightarrow \dots \\
	FOLLOW_1(\texttt{e}) & = \{ \#, \texttt{v} \} ~~~~~~~ S \Rightarrow \texttt{e}F \Rightarrow \texttt{e}\underline{~~} \text{ ~ és ~ } S \Rightarrow \texttt{e}F \Rightarrow \texttt{e}\underline{\texttt{v}}\texttt{e}F
\end{flalign*}

Az elágazások feltételeinek meghatározását a következőképp tehetjük meg.

\begin{itemize}
	\item Az $\prodrule{A}{\alpha}$ szabályhoz meghatározzuk a $FIRST_1(\alpha)$ halmazt.
	\item Ha ebben van $\emptyword$, akkor kivesszük $\emptyword$-t és helyette hozzávesszük a halmazhoz $FOLLOW_1(A)$ elemeit.
	\item Az így kapott halmaz elemeiből (pl. $x_1$, $x_2$, ..., $x_n$) képezzük az elágazás feltételét:
	\[ next = x_1 \lor next = x_2 \lor \dots \lor next = x_n. \]
\end{itemize}

\pagebreak

\begin{figure}[h!]
	\begin{stuki*}{conditions$(\prodrule{A}{\alpha} : P) : T\{\}$}
		\stm{X := FIRST_1(\alpha)}
		\begin{IF}[70]{2}{\stm{\emptyword \in X}}
			\stm{X := X \setminus \{\emptyword\}}
			\stm{X := X \cup FOLLOW_1(A)}
			\ELSE
			\stm*{SKIP}
		\end{IF}
		\stm{\textbf{return } X}
	\end{stuki*}
	\caption{Az elágazás feltételének meghatározásának algoritmusa}
\end{figure}

Ellenőrizhető a struktogram segítségével, hogy valóban ezen eredmények jönnek ki.
\begin{flalign*}
	conditions(\prodrule{S}{\emptyword}) & = \{ \# \} \\
	conditions(\prodrule{S}{\texttt{e}F}) & = \{ \texttt{e} \} \\
	conditions(\prodrule{F}{\emptyword}) & = \{ \# \} \\
	conditions(\prodrule{F}{\texttt{ve}F}) & = \{ \texttt{v} \}
\end{flalign*}

\begin{tcolorbox}
	\begin{theorem}[$LL(1)$ tulajdonság ellenőrzése]
		Egy környezetfüggetlen nyelvtan pontosan akkor $LL(1)$ tulajdonságú, ha bármely két $\prodrule{A}{\alpha}$, $\prodrule{A}{\beta}$ (a két $A$ megegyezik!) szabályokhoz a fenti módon meghatározott halmazok diszjunktak. 
	\end{theorem}
\end{tcolorbox}

\begin{itemize}
	\item Példa $LL(1)$ tulajdonságú nyelvtan.
	\begin{flalign*}
		conditions(\prodrule{S}{\emptyword}) & = \{ \# \} \\
		conditions(\prodrule{S}{\texttt{e}F}) & = \{ \texttt{e} \} \\
		\{\#\} \cup \{\texttt{e}\} & = \emptyset ~~~ \checkmark \\\\
		conditions(\prodrule{F}{\emptyword}) & = \{ \# \} \\
		conditions(\prodrule{F}{\texttt{ve}F}) & = \{ \texttt{v} \} \\
		\{\#\} \cup \{\texttt{v}\} & = \emptyset ~~~ \checkmark
	\end{flalign*}
	\item Példa nem $LL(1)$ tulajdonságú nyelvtanra.
	\begin{flalign*}
		conditions(\prodrule{S}{\emptyword}) & = \{ \# \} \\
		conditions(\prodrule{S}{N}) & = \{ \texttt{e} \} \\
		\{\#\} \cup \{\emptyword\} & = \emptyset ~~~ \checkmark \\\\
		conditions(\prodrule{N}{\texttt{e}}) & = \{ \texttt{e} \} \\
		conditions(\prodrule{N}{N\texttt{ve}}) & = \{ \texttt{e} \} \\
		\{\texttt{e}\} \cup \{\texttt{e}\} & \neq \emptyset ~~~ \lightning
	\end{flalign*}
\end{itemize}

\newpage

\section{Alulról felfele elemzés}

Az \textbf{alulról felfele elemzés} a másik lehetséges stratégiája a szintaktikus elemzésnek. A \textbf{terminálisokból a startszimbólum felé} építjük a szintaxisfát. A \textbf{bemenet feldolgozása} továbbra is \textbf{balról jobbra} történik, azonban az elemzés a \textbf{legjobboldalibb levezetés inverzé}t állítja elő -- a legjobboldalibb levezetés azt jelenti, hogy több terminális esetén a legjobboldalibbat helyettesíti.

Szemléltessük az korábbi nyelvtanon:
\begin{flalign*}
	& \prodrule{S}{AB} \\
	& \prodrule{A}{a} \\
	& \prodrule{B}{bc}.
\end{flalign*}
Legyen a bemeneti szövegünk továbbra is: $abc$. A szó szintaxisfáját így kapjuk meg:

\begin{figure}[h!]
	\centering
	\begin{minipage}{0.2\linewidth}
		\begin{center}
			\begin{forest}
				for tree={ edge={draw=none}}
				[$ $
				[$ $
				[$a$, tier=terminal]
				]
				[$ $
				[$b$, tier=terminal]
				[$c$, tier=terminal]
				]
				]
			\end{forest}
			
			$abc$
		\end{center}
	\end{minipage}
	\begin{minipage}{0.2\linewidth}
		\begin{center}
			\begin{forest}
				for tree={ edge={-}}
				[$ $, edge={draw=none}
				[$A$, edge={draw=none}
				[$a$, tier=terminal]
				]
				[$ $, edge={draw=none}
				[$b$, tier=terminal, edge={draw=none}]
				[$c$, tier=terminal, edge={draw=none}]
				]
				]
			\end{forest}
			
			$abc \Leftarrow Abc$
		\end{center}
	\end{minipage}
	\begin{minipage}{0.2\linewidth}
		\begin{center}
			\begin{forest}
				for tree={ edge={-}}
				[$ $, edge={draw=none}
				[$A$, edge={draw=none}
				[$a$, tier=terminal]
				]
				[$B$, edge={draw=none}
				[$b$, tier=terminal]
				[$c$, tier=terminal]
				]
				]
			\end{forest}
			
			$abc \Leftarrow Abc \Leftarrow AB$
		\end{center}
	\end{minipage}
	\begin{minipage}{0.30\linewidth}
		\begin{center}
			\begin{forest}
				for tree={ edge={-}}
				[$S$
				[$A$
				[$a$, tier=terminal]
				]
				[$B$
				[$b$, tier=terminal]
				[$c$, tier=terminal]
				]
				]
			\end{forest}
			
			$abc \Leftarrow Abc \Leftarrow AB \Leftarrow S$
		\end{center}
	\end{minipage}
	\caption{Alulról felfele elemzés lépései}
\end{figure}

Az alulról felfele elemzők egyik gyakori változatával, az ún. LR elemzőkkel fogunk megismerkedni. A pontos definícióját a későbbiekben kimondjuk.

Hasonlóan az $LL$-hez, az $LR$-elemzés is \textbf{verem alapú}, ám a verem nem futás idejű -- tehát a kódban valóban példányosítanunk kell egyet. \textbf{Ebben gyűjtjük a szimbólumokat} (terminálisokat és nemterminálisokat egyaránt) egészen addig, amíg a megfelelő szabályjobboldal megjelenik benne.

Tartozik hozzá \textbf{két művelet}.
\begin{enumerate}
	\item \underline{Léptetés (\textit{shift} vagy \textit{push})}: A következő token elhelyezése a verem tetején.
	\item \underline{Redukció (\textit{reduce} vagy \textit{pop})}: A szabályjobboldal helyettesítése
	szabálybaloldallal a veremben, közben a szintaxisfa bővítése.
\end{enumerate}

Szemléltessük a műveleteket a kövektező \textit{balrekurzív} nyelvtanon (ami szintén a vesszővel elválasztott listák nyelét fejezi ki):
\begin{flalign*}
	& \prodrule{S}{\emptyword \mid N} \\
	& \prodrule{N}{\texttt{e} \mid N\texttt{ve}}
\end{flalign*}
A példaszavunk továbbra is legyen az ``\texttt{eveve}''.

Kezdetben a vermünk üres, így előreolvasunk egy karaktert. Betesszük a verembe (\texttt{e}) -- azaz léptetünk. Ekkor megjelent egy szabályjobboldal ($\prodrule{N}{\texttt{e}}$), amit kicserélhetünk a bal oldalával ($N$).
\[ \# \genword{}{shift} \texttt{e} \genword{}{reduce} N. \]
Folytatjuk a léptetést. A verem állapota így $N\texttt{v}$ lesz. Nincs ilyen alakú szabályjobboldal, így újból léptetünk. A veremben $N\texttt{ve}$ lesz. Ez már helyettesíthető szabálybaloldalra ($\prodrule{N}{N\texttt{ve}}$), így redukálunk (verem: $N$). Kettőt léptetünk (verem: $N\texttt{v}$, $N\texttt{ve}$), majd redukálunk (verem: $N$). Mivel elfogytak a beolvasandó karaktereink, így tovább redukálhatjuk a verem tartalmát. Az elemzés akkor sikeres, ha csupán az $S$ marad benne a legvégén.
\[ \# \genword{}{shift} \texttt{e} \genword{}{red.} N \genword{}{shift} N\texttt{v} \genword{}{shift} N\texttt{ve} \genword{}{red.} N \genword{}{shift} N\texttt{v} \genword{}{shift} N\texttt{ve} \genword{}{red.} N \genword{}{red.} S. \]
\begin{figure}[h!]
	\centering
\begin{forest}
	for tree={edge={-}}
	[$S$
		[$N$
			[$N$
				[$N$
					[\texttt{e}, tier=terminal]
				]
				[\texttt{v}, tier=terminal]
				[\texttt{e}, tier=terminal]
			]
			[\texttt{v}, tier=terminal]
			[\texttt{e}, tier=terminal]
		]
	]
\end{forest}
	\caption{Az $LR$ elemzés által létrehozott szintaxisfa}
\end{figure}

Ha visszapillantunk a korábbi ábrára, ahol egy léptetés és redukció után az $N$ szerepelt a veremben, észrevehetjük, hogy akár rögtön abban a lépésben is redukálhattuk volna $S$-re a tartalmát -- ezzel kihagyva a szövegünk hátralévő 80\%-át.

A korábbiakhoz hasonlóan, felmerülhet a kérdés: \textbf{hogyan döntjük el, hogy mikor melyik műveletet végezzük el}?
Ennek az eldöntéséhez figyelembe kell vennünk a \textit{következő valahány token}t (ami \textbf{előreolvasás}t jelent), valamint az \textbf{elemző állapotát} (ami a verembe bekerülő szimbólumoktól függően változik).

Ezzel el is érkeztünk ahhoz, hogy kimondjuk az \framebox{$LR(k)$ nyelvtan} pontos definícióját.

\begin{tcolorbox}
	\begin{definition}[$LR(k)$ nyelvtan]
		Egy környezetfüggetlen nyelvtan $LR(k)$ tulajdonságú valamely $k \in \mathbb{N}$
		számra, ha az elemzés pillanatnyi állapotából és legfeljebb $k$ token
		előreolvasásával egyértelműen meghatározható, hogy léptetés vagy
		redukció következik, és redukció esetén az alkalmazandó nyelvtani szabály
		is kiderül.
	\end{definition}
\end{tcolorbox}

Az elnevezés a ``\textit{\textbf{l}eft to right using \textbf{r}ightmost derivation}'' elnevezés angol rövidítéséből származik.

Az $LR(1)$ elemzést fogjuk részletesen megvizsgálni -- egyetlen szimbólum előreolvasása elegendő. 

A korábbi megállapításaink alapján létre kell hoznunk egy \textbf{elemző táblázat}ot, ami meghatározza a lépéseket és az állapotátmeneteket. Mivel az $LR$ elemzés verem alapú, így ez is egy \textbf{veremautomata} lesz. Négy \textbf{akció} szerepel egy ilyen táblázatban: \textit{léptetés}, \textit{redukció}, \textit{elfogadás} és \textit{hiba}.

\newpage

\begin{figure}[h!]
	\centering
	\begin{tabular}{|c||c|c|c|c|}
		\hline
		& \texttt{e} & \texttt{v} & input vége & $N$ \\
		\hline\hline
		0 & léptetés : 2 & hiba & elfogadás & 1 \\
		\hline
		1 & hiba & léptetés : 3 & elfogadás & hiba \\
		\hline
		2 & hiba & redukció : $\prodrule{N}{\texttt{e}}$ & redukció : $\prodrule{N}{\texttt{e}}$ & hiba \\
		\hline
		3 & léptetés : 4 & hiba & hiba & hiba \\
		\hline
		4 & hiba & redukció : $\prodrule{N}{N\texttt{ve}}$ & redukció : $\prodrule{N}{N\texttt{ve}}$ & hiba \\
		\hline
	\end{tabular}
	\caption{A nyelvtanunk $LR(1)$ elemző táblázata}
\end{figure}

Talán nem meglepő, a nyelvtannak ezen tulajdonságának ellenőrzésére is létezik tétel.

\begin{tcolorbox}
	\begin{theorem}[$LR(1)$ tulajdonság ellenőrzése]
		Egy környezetfüggetlen nyelvtan pontosan akkor $LR(1)$ tulajdonságú, ha az
		elemző táblázatot kitöltő algoritmus konfliktusmentesen kitölti a táblázatot.
	\end{theorem}
\end{tcolorbox}

\textbf{\textit{Megjegyzés.}} A táblázat a nyelvtanból algoritmikusan létrehozható, de nem része a tananyagnak. Állítólag elég bonyolult.

