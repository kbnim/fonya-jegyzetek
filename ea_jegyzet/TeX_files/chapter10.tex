\chapter{Az Assembly alapjai}

Az \textbf{Assembly} egy \textit{alacsony szintű programozási nyelv}, amely segít áthidalni a masag szintű nyelvek és a gépi kód közti hatalmas ``szakadékot''. Valójában a gépi kódnak egy ember számára olvashatóbb változatáról van szól, ami a rejtélyes hexadecimális számok helyett korlátozott számú rövid, tömör nevű \textbf{műveltek}et, valamint \textbf{regiszterek}et használ. Megengedi a programozó számára, hogy a memóriacímek azonosítására \textbf{címkék}et használjunk.

Az Assembly nem egy egységes nyelv, hanem inkább egy nyelvcsalád, aminek a nyelvei architektúránként eltér. Azonban vannak közös jellemzői. Fordítóprogramját \textbf{Assembler}nek nevezzük. A tanulmányai folyamán a \textbf{32-bites}, x\textbf{86-os architektúrá}jú, NASM szintaxisú Assemblyvel fogunk foglalkozni.

\begin{figure}[h]
	\centering
\begin{minipage}{0.05\linewidth}
	~
\end{minipage}
\begin{minipage}{0.25\linewidth}
	\begin{lstlisting}[style=cppstyle]
int sum = 0;
for (int i = 0; 
	 i < len; 
	 ++i)
{
	sum += t[i];
}
	\end{lstlisting}
\end{minipage}
\begin{minipage}{0.03\linewidth}
	~
\end{minipage}
\begin{minipage}{0.25\linewidth}
	\begin{lstlisting}[style=asmstyle]
mov ecx,0
mov eax,0
eleje:
cmp ecx,10
jge vege
add eax,[ebx+4*ecx]
inc ecx
jmp eleje
vege:
	\end{lstlisting}
\end{minipage}
\begin{minipage}{0.03\linewidth}
	~
\end{minipage}
\begin{minipage}{0.25\linewidth}
	\begin{lstlisting}[style=machinestyle]
B9 00 00 00 00
B8 00 00 00 00
81 F9 0A 00 00 00
7D 06
03 04 8B
41
EB F2
	\end{lstlisting}
\end{minipage}
\caption{C++ kód, Assembly kód és gépi kód}
\end{figure}

\begin{figure}[h]
	\centering
	\begin{tikzpicture}
		% High-level source code
		\node[block, minimum width=5cm, minimum height=1.5cm] (highlevel) at (0,0) {};
		\node[align=center] at (highlevel) {magas-szintű\\forráskód};
		
		% Compiler
		\node[block, minimum width=5cm, minimum height=1cm] (compiler) at (0,-2.0) {};
		\node[align=center] at (compiler) {Assembly kód};
		
		% Assembly source code
		\node[block, minimum width=5cm, minimum height=1cm] (assembly) at (0,-4) {};
		\node[align=center] at (assembly) {gépi kód};
		
		% Arrows
		\draw[->, thick] (highlevel.south) -- (compiler.north) node[midway, right] {compiler};
		\draw[->, thick] (compiler.south) -- (assembly.north) node[midway, right] {assembler};
	\end{tikzpicture}
	\caption{A forráskód állapotának szakaszai}
\end{figure}

\newpage

\section{Adattárolás}

A számítógépen alapvetően háromféleképpen tárolhatjuk az adatainkat.

Léteznek a processzorban a \framebox{\textbf{regiszter}ek} , amik nagyon \textbf{gyorsak}, bár \textbf{kis méretű} adatok tárolására képesek. Ezt a kapacitást az adott architektúra határozza meg, így egy \textbf{32-bites processzoron 1 regiszter 32 bitet} (= 4 bájtot) tárol.\footnote{Vagy általánosan $n$-bites processzorban $n$-bitesek a regiszterek, ahol $n\in \mathbb{N}^+$ 2 hatványa.} Azon aktuális adatokat tároljuk el bennük, melyekkel \textbf{műveleteket akarunk végezni} (\textit{hatékonyan}).

A \framebox{\textbf{memória}} nevezhető az ``arany középútnak'', ugyanis \textbf{közepesen gyors} és \textbf{közepes a tárkapacitása}. Praktikus a programkód és a változók tárolására. A statikus memória, a \textbf{stack} és a \textbf{heap} is ezen helyezkedik el. A stack a program futásidejű, veremszerkezetű memóriája. A heapről sajnos nem lesz szó a tantárgy keretein belül.

És végül, de nem utolsó sorban, említésre méltóak a \framebox{\textbf{háttértárak}} . Ezek összehasonlítva \textbf{lassú}ak, cserébe \textbf{óriási tárkapacitás}sal rendelkeznek.

\subsection{Regiszterek}

32-bites architektúrán a regiszterek nevei \texttt{e}-vel kezdődnek.

\begin{figure}[h]
	\centering
	\begin{tabular}{|l|l|}
		\hline\hline
		\multicolumn{2}{|c|}{\textbf{Általános célú regiszterek}} \\
		\hline
		\makecell[l]{\texttt{eax}, \texttt{ebx}, \\ \texttt{ecx}, \texttt{edx}, \\ \texttt{esi}, \texttt{edi}} &  \makecell[l]{Egymással felcserélhetően használhatók (többnyire). \\
		Néhány \textit{konvenció}, ha C függvények hívásához használjuk őket. \\
		-- A függvény csak az \texttt{eax}, \texttt{ecx} és \texttt{edx} regisztereket hagyja \\
		~~ változatlanul, a többit ``elronthatja''. \\
		-- A visszatérési érték az \texttt{eax} regiszterbe kerül.
		}	\\
		\hline\hline
		\multicolumn{2}{|c|}{\textbf{Veremmel kapcsolatos regiszterek}} \\
		\hline
		\texttt{esp} & \textit{Stack pointe}r, a futási idejű verem tetejét mutatja. \\
		\hline
		\texttt{ebp} & \makecell[l]{\textit{Base pointer}, az éppen aktív eljáráshoz/függvényhez \\ tartozó adatokra mutat a veremben.} \\
		\hline\hline
		\multicolumn{2}{|c|}{\textbf{Egyéb regiszterek}} \\
		\hline
		\texttt{eip} & \textit{Instruction pointer}, a következő végrehajtandó utasításra mutat. \\
		\hline
		\texttt{eflags} & \makecell[l]{Jelzőbitek gyűjteménye, pl. ``az előző aritmetikai utasítás \\ eredménye nulla volt-e?''.} \\
		\hline
	\end{tabular}
	\caption{Regiszterek csoportosítása}
\end{figure}

Az \textit{egyéb regisztereket} \textbf{nem érjük el közvetlenül}, egyes utasítások olvassák/írják őket (pl. a \texttt{cmp} a \texttt{eflags}-et vagy a \texttt{call} a \texttt{eip}-t).

A regiszterek szerkezete a következő. Az általános célú regiszterek (pl. az \texttt{eax}, \texttt{ebx}, \texttt{ecx}, \texttt{edx}) feloszhatók alsóbb szeletekre, akár több szinten is. Így lesz egy 16-bites \texttt{ax} (a 0.-től a 15. bitig) és egy szintén 16-bites, ám név nélküli szelete (a 16.-tól a 31. bitjéig). Az \texttt{ax} tovább osztható \texttt{al}-re és \texttt{ah}-ra (8-8 bitesek, a \textit{low half} és \textit{high half} elnevezésből). Az \texttt{eax}, \texttt{ax}, \texttt{al} mind egy-egy regiszter, de \textbf{nem függetlenek egymástól}. Ugyanis ha az \texttt{al} megváltozik, akkor az \texttt{ax} és az \texttt{eax} is vele változik.

\begin{figure}[h]
	\centering
	\begin{tikzpicture}
		\node[block, minimum width=12cm, minimum height=1cm] (eax) {\texttt{eax} (32 bit)};
		\node[block, minimum width=6cm, minimum height=1cm] at (3,-1) {\texttt{ax} (16 bit)};
		\node[selection, minimum width=6cm, minimum height=1cm] at (-3,-1) {(\textit{16 bit})};
		\node[block, minimum width=3cm, minimum height=1cm] at (4.5,-2) {\texttt{al} (8 bit)};
		\node[block, minimum width=3cm, minimum height=1cm] at (1.5,-2) {\texttt{ah} (8 bit)};
		
		\draw[-] (6, -3)  -- (6, 0) {};
		\draw[-] (-6, -3) -- (-6, 0) {};
		
		\node at (6, -3.2)  {0. bit};
		\node at (-6, -3.2) {31. bit};
	\end{tikzpicture}
	\caption{Az \texttt{eax}, \texttt{ebx}, \texttt{ecx}, \texttt{edx} szerkezete}
\end{figure}

Más regiszterek (\texttt{esp}, \texttt{esi}, \texttt{edi}, \texttt{ebp}, \texttt{eip}) is feloszhtatók ugyan kevesebb, de hasonlóan ``nevezetes szeletekre'', pl. \texttt{esp} (32-bit) -- \texttt{sp} (alsó 16 bit).

\begin{figure}[h]
	\centering
	\begin{tikzpicture}
		\node[block, minimum width=12cm, minimum height=1cm] (esp) {\texttt{esp} (32 bit)};
		\node[block, minimum width=6cm, minimum height=1cm] at (3,-1) {\texttt{sp} (16 bit)};
		\node[selection, minimum width=6cm, minimum height=1cm] at (-3,-1) {(\textit{16 bit})};
		
		\draw[-] (6, -2)  -- (6, 0) {};
		\draw[-] (-6, -2) -- (-6, 0) {};
		
		\node at (6, -2.2)  {0. bit};
		\node at (-6, -2.2) {31. bit};
	\end{tikzpicture}
	\caption{Az \texttt{esp}, \texttt{esi}, \texttt{edi}, \texttt{ebp}, \texttt{eip} szerkezete}
\end{figure}


\subsection{Címkék}

A \textbf{címkék} arra szolgálnak, hogy az egyes memóriacímekre kényelmesebben tudjunk hivatkozni. Önmagában nem tárolnak sem típusinformációt, sem a változó méretét. Az utóbbiról kifejezetten fontos, hogy gondoskodjunk, ugyanis csak az alapján képes eldönteni a program, hogy az adott memóriacímtől kezdve hány bájtot olvasson be. Ha belegondolunk, ez hasonlít arra, ahogyan a C-ben a tömbök, pointerek működnek.

\begin{lstlisting}[style=asmstyle, caption={Assembly program. A forráskód kiterjesztése \texttt{*.asm}}]
global main		; global label used to denote the entry point of the program
extern <label>  ; label is declared but defined elsewhere (externally)

; uninitialised variables
section .bss
a: resd 1    ; 1x4 bytes reserved at 'a'
b: resw 1    ; 1x2 bytes reserved at 'b'
c: resb 256  ; 256x1 bytes reserved at 'c' (i.e. character array)

; initialised variables
section .data
d: dd 42      ; 1x4 bytes defined as '42' at 'd'
e: dw 1,2,3,4 ; 4x2 bytes defined as '1,2,3,4' at 'e' (array)
f: db 'a'     ; 1x1 bytes defined as 'a' (ASCII) at 'f'

; source code starts here
section .text
main:
	; machine instructions ...
\end{lstlisting}

\begin{itemize}
	\item \asmexample{global main} : Ebben a fájlban definiáljuk a ``main'' címkét, és szeretnénk, hogy globálisan látható
	legyen.
	\item \asmexample{extern <label>} : A \texttt{<label>} címke máshol van definiálva, de itt
	szeretnénk használni.
	\item \asmexample{section .bss} : Kezdőérték nélküli memóriaterület (``\textit{block starting symbol}''). \\
	A rövidítések jelentése: \texttt{res} $\longrightarrow$ ``\textit{reserve}''.
	\begin{flalign*}
		\texttt{res} \begin{cases}
			\texttt{b} \longrightarrow \text{``\textit{byte}''} & = 1 ~ \text{bájt}\\
			\texttt{w} \longrightarrow \text{``\textit{word}''} & = 2 ~ \text{bájt} \\
			\texttt{d} \longrightarrow \text{``\textit{double}''} & = 4 ~ \text{bájt}.
		\end{cases}
	\end{flalign*}
	\item \asmexample{section .data} : Kezdőértékkel rendelkező memóriaterület. \\
	A rövidítések jelentése: \texttt{d} $\longrightarrow$ ``\textit{define}''.
	\begin{flalign*}
		\texttt{d} \begin{cases}
			\texttt{b} \longrightarrow \text{``\textit{byte}''} & = 1 ~ \text{bájt}\\
			\texttt{w} \longrightarrow \text{``\textit{word}''} & = 2 ~ \text{bájt} \\
			\texttt{d} \longrightarrow \text{``\textit{double}''} & = 4 ~ \text{bájt}.
		\end{cases}
	\end{flalign*}
	\item \asmexample{section .text} : Itt kezdődik a programkód, az utasítások sorozata.
	\begin{itemize}
		\item Minden utasításnak lehet címkéje, de nem kötelező.
		Lehet csak címkét tartalmazó sor is.
		\item Önálló assembly programoknál a program belépési pontja a \texttt{\_start} címke,
		de megírhatjuk egy C program main függvényét is, ekkor \texttt{main} címkét
		használunk.
		\item Az utasításoknak van neve (mnemonikja) és operandusai.
		\item Megjegyzések: a \texttt{;} karaktertől a sor végéig.
	\end{itemize}
	\item \asmexample{main:} : Main címke; itt kezdődik a program ‘main’ függvénye.
\end{itemize}

A memóriahivatkozás címkékkel a következőképp történik:
\[ \texttt{<size> [<label>]}, \] ahol a \texttt{<size>} lehet \texttt{byte} (1 bájt), \texttt{word} (2 bájt) vagy \texttt{dword} (4 bájt), a \texttt{<label>} meg egy címke. Fontos, hogy ugyanolyan mérettel ``paraméterezzük fel'', amilyennek deklaráltuk.

A kódban az alábbi módon történne a memóriahivatkozás:
\[ \texttt{dword [a]}, ~~~~~ \texttt{word [b]}, ~~~~~ \texttt{byte [c]} ~~ \text{(a tömb első eleme)}. \]
A hivatkozásba írhatók regiszterekből, címkékből és számliterálokból álló egyes kifejezések, pl.:
\[ \texttt{dword [a+4*ecx]} ~~ \text{(az \texttt{a} tömb 4. eleme, ha 4 bájtosak az egyes elemek)}. \]

\pagebreak

\section{Utasítások, műveletek}

Először általánosságokban beszélünk az Assembly utasításairól, utána csoportonként átnézzük a specifikusabb részleteket.

Ahogy azt megfigyelhettük, az Assembly utasításai \textbf{prefix jelölés}t használnak. Ez jó, mert a fordítók hatékonyan fel tudják dolgozni (lásd \textit{Algoritmusok és adatszerkezetek I.}). Egy utasítás felvehet 0, 1 vagy 2 paramétert.

A kétparaméteres utasítások szerkezete a következő:
\[ \texttt{instruction <destination> <source>}. \]

Értelemszerűen az \texttt{instruction} jelöli az utasítást. A \texttt{<destination>} lehet regiszter vagy memóriahivatkozás, míg a \texttt{<source>} lehet regiszter, memóriahivatkozás vagy literál. 

\textbf{Legfeljebb az egyik lehet memóriahivatkozás!}

Továbbá biztosítanunk kell a megfelelő \textbf{offset}et, azaz \textbf{memóriahivatkozásnál meg kell adnunk az olvasandó méretet }a fent bemutatott módon (\texttt{byte}, \texttt{word}, \texttt{dword}). Ha valamelyik operandus regiszter, akkor abból kiderül, így elhagyható.

A művelet \textbf{végeredménye a \texttt{<destination>} által jelölt memóriacímbe lesz írva}.

\begin{lstlisting}[style=asmstyle, caption={Helyes paraméterezés}]
; correct
mov eax,0   ; one of them is a register, the size is evident
mov bx,ax   ; the sizes of the two registers are equal
mov [lab1], al ; one of them is a register, the size is evident
mov dword [lab2], 987 ; the size of the label needs to be specified
mov ebx, [lab3] ; one of them is a register, the size is evident
\end{lstlisting}

\begin{lstlisting}[style=asmstyle, caption={Helytelen paraméterezés}]
; incorrect
mov byte [lab4], byte [lab5] ; two labels are not allowed
mov al, ax ; the sizes of the two registers are NOT the same
\end{lstlisting}

Egyparaméteres utasításoknál a végeredmény az egyetlen paraméter memóriacímében lesz elhelyezve:
\[ \texttt{instruction <parameter>}. \]
A \texttt{<parameter>} operandus lehet regiszter vagy memóriahivatkozás.

\subsection{Adatmozgató utasítások}

Utasítás: \texttt{mov}. A mozgatás valójában másolást jelent: a ``honnan'' nem változik meg.

\subsection{Aritmetikai utasítások}

Kétparaméteres műveletek: \texttt{add} (összeadás), \texttt{sub} (kivonás).

Egyparaméteres műveletek: \texttt{inc} (inkrementálás), \texttt{dec} (dekrementálás).

A szorzás (\texttt{mul}) és osztás (\texttt{div}) igs egyparaméteres műveletek, azonban a működésük nem teljesen intuitív.

A szorzás összeszorzza az \texttt{eax} és a paraméterül kapott memóriacím tartalmát. Az eredmény az \texttt{eax}-be kerül, azonban túlcsoldulás esetén a további bitek az \texttt{edx}-ben lesznek rögzítve.

\begin{lstlisting}[style=asmstyle]
mov eax, 5      ; Load 5 into eax
mov ebx, 6      ; Load 6 into ebx
mul ebx         ; Multiply eax by ebx (5 * 6 = 30)
; After this, eax = 30 and edx = 0 because the result fits in 32 bits
\end{lstlisting}

A \texttt{div} is hasonló elvet követ -- ott csupán nem a túlcsordulás esete áll fenn, hanem az osztási maradéké. Ez az, amit az \texttt{edx}-ben eltárol.

\begin{lstlisting}[style=asmstyle]
mov eax, 30     ; Load 30 into eax
mov edx, 0      ; Clear edx
mov ecx, 4      ; Load 4 into ecx
div ecx         ; Divide edx:eax by ecx (30 / 4)
; After this, eax = 7 (quotient), edx = 2 (remainder)
\end{lstlisting}

\textbf{\textit{Megjegyzés}}. Felhívjuk a figyelmet, hogy a \texttt{mul} és \texttt{div} előjel nélküli egész számoknak tekinti az operandusait. Előjeles számok esetén \texttt{imul} és \texttt{idiv} használandó.

\subsection{Bitműveletek}

Kétparaméteres műveletek: \texttt{and}, \texttt{or}, \texttt{xor}.

Egyparaméteres műveletek: \texttt{not}.

\subsection{Ugróutasítások}

Módosíthatják azon regiszterek tartalmát (\texttt{eip}\footnote{A következő végrehajtandó utasításra mutat.}, \texttt{eflags}\footnote{Jelzőbitek gyűjteménye, pl. ``az előző aritmetikai utasítás eredménye nulla volt-e?''.}), amihez amúgy nincs közvetlen hozzáférésünk.

\subsubsection{Feltétel nélküli ugrás}

Egyparaméteres utasítások: \texttt{jmp <label>}.\footnote{Megfeleltethető ``\textit{az átkos}'' \texttt{goto} utasításnak.} Módosítja az \texttt{eip}-t.%Általában címkékkel szoktuk felparaméterezni.

// Ábra

\subsubsection{Feltételes ugrások}

Kétparaméteres utasítások: \texttt{cmp <mit> <mivel>} (\textit{compare}).
\begin{enumerate}
	\item A \texttt{cmp} utasítás kivonást végez, de nem tárolja el az eredményt, hanem annak előjele alapján jelzőbiteket állít át az \texttt{eflags} regiszterben.
	\item Ha a kivonás eredménye 0 (azaz az összehasonlított értékek egyenlők), akkor a \textbf{zero flag} 1 lesz, különben 0.
\end{enumerate}

Egyparaméteres utasítások: \texttt{je} (\textit{jump if equal}), \texttt{jne} (\textit{jump if not equal}), \texttt{jb} (\textit{below}) = \texttt{jnae}, \texttt{ja} (\textit{above}) = \texttt{jnbe}, \texttt{jnb} = \texttt{jae}, \texttt{jna} = \texttt{jbe}. 
\begin{enumerate}
	\item A feltételes ugró utasítások a jelzőbiteket figyelik.
	\item A \texttt{je} akkor ugrik, ha a \textbf{zero flag} 1, egyébként a következő utasításra kerül a vezérlés.
\end{enumerate}

\textbf{\textit{Megjegyzés}}. Ezek előjel nélküli egész számokat feltételeznek!

// Ábra

\subsubsection{Feltételes adatmozgatás}

Kétparaméteres utasítások: \texttt{cmov <hova> <honnan>} (\textit{conditional move}).

// Ábra

\subsubsection{Elágazások, ciklusok}

\begin{figure}[h]
	\centering
	\begin{minipage}{0.25\linewidth}
		\begin{lstlisting}[style=asmstyle, caption={Egyágú elágazás}]
cmp eax,ebx
je egyenlo
mov eax,ebx
egyenlo:
		\end{lstlisting}
	\end{minipage}
\begin{minipage}{0.05\linewidth}
	~
\end{minipage}
\begin{minipage}{0.25\linewidth}
	\begin{lstlisting}[style=asmstyle, caption={Kétágú elágazás}]
cmp ebx,ecx
ja nagyobb
mov eax,ecx
jmp vege
nagyobb:
mov eax,ebx
vege:
	\end{lstlisting}
\end{minipage}
\begin{minipage}{0.05\linewidth}
	~
\end{minipage}
\begin{minipage}{0.25\linewidth}
	\begin{lstlisting}[style=asmstyle, caption={Ciklus}]
mov eax,0
eleje: cmp ecx,0
je vege
add eax,ecx
dec ecx
jmp eleje
vege:
	\end{lstlisting}
\end{minipage}
\end{figure}

\subsection{Veremműveletek}

Minden futó programhoz tartozik egy verem vagy más néven \textbf{stack}. Ezt a memóriaterületet az \textbf{operációs rendszer rendeli hozzá} a programhoz. Az \asmexample{esp} regiszter mutatja a verem tetejét. Általában itt tároljuk:
\begin{itemize}
	\item függvények paraméterei
	\item függvények visszatérési címe
	\item függvények lokális változói
	\item időlegesen tárolt adatok
\end{itemize}

Nem meglepően, a kapcsolódó \textit{egyparaméteres műveletei}: \texttt{push} és \texttt{pop}.

\textbf{\textit{Megjegyzés}}. Mindkét esetben csak 2 vagy 4 bájtos lehet az operandusa a két utasításnak!

// Ábra

\subsubsection{Függvényhívás és visszatérés}

Egyparaméteres utasítás: \texttt{call <címke>}. Valójában ez is egyfajta ugróutasítás.

Paraméter nélküli utasítás: \texttt{ret}.

\textbf{\textit{Megjegyzés}}. Mindkettő módosítja az \texttt{eip} és az \texttt{esp} értékét!

// Ábra

\subsubsection{Függvényhívás paraméterrel}

Nincsenek külön műveletek erre -- az idáig bemutatottakat alkalmazzuk.
\begin{itemize}
	\item A függvény paraméterét a
	verembe tesszük a hívás
	előtt.
	\item A függvény törzse
	kimásolja a paramétert a
	veremből.
	\item A visszatérési érték az
	\texttt{eax} regiszterben van.
	\item A hívás után kivesszük a
	paramétert (vagy legalább
	a veremmutatót
	visszaállítjuk).
\end{itemize}

%\newpage

\section{Fordítás}

A fordításhoz a \texttt{nasm} assemblert használjuk, ami előállítja a \textbf{tárgykódot}.

Vegyük az alábbi kódokat!

\begin{figure}[h]
\begin{minipage}{0.45\linewidth}
	\begin{lstlisting}[style=asmstyle, caption={\texttt{addone.asm}}]
global main
extern write_natural
extern read_natural

section .text
main:
call read_natural
inc eax
push eax
call write_natural
add esp,4
mov eax,0
ret
	\end{lstlisting}
\end{minipage}
\begin{minipage}{0.05\linewidth}
	~
\end{minipage}
\begin{minipage}{0.45\linewidth}
	\begin{lstlisting}[style=cppstyle, caption={\texttt{io.c}}]
#include <stdio.h>

void write_natural(unsigned n) 
{
	printf("%u\n", n);
}

unsigned read_natural() 
{
	unsigned ret;
	scanf("%u", &ret);
	return ret;
}
	\end{lstlisting}
\end{minipage}
\end{figure}

Az \texttt{addone.asm} meghívja az \texttt{io.c}-ből a \texttt{read\_natural} függvényt, ami beolvas egy előjel nélküli természetes számot, majd ezt kiíratja a \texttt{write\_natural} meghívásával. Végül a program terminál.

Az említett két függvényt az egyszerűség kedvéért C-ben írjuk meg.
Lehetne tisztán Assemblyben is, de az jóval bonyolultabb volna.

Első lépésként az Assembly kódot fordítjuk le. A \texttt{-felf} kapcsolóval megadjuk a fájl formátumát (a mi esetünkben ez \texttt{elf}).

\begin{lstlisting}[style=machinestyle]
nasm -felf addone.asm
\end{lstlisting}

Ezután a C fordítóval lefordítjuk gépi kódra az \texttt{io.c}-t, majd összelinkeljük a kapott tárgykódokat, ügyelve arra, hogy 32-bites architektúrájúra fordítsuk le (\texttt{-m32}).

\begin{lstlisting}[style=machinestyle]
gcc -m32 -oaddone io.c addone.o
\end{lstlisting}

Végül nincs más hátra, mint hogy kipróbáljuk!

\begin{lstlisting}[style=machinestyle]
./addone
41
42
\end{lstlisting}