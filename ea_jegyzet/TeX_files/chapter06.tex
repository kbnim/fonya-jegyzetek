\chapter{Fordítóprogramok}

A jegyzet első felében részletezett elméleti háttérrel felvértezve már képesek vagyunk nyelveket definiálni. A most következő részben betekintést nyerhetünk abba, hogy miként lesz a nyelvünkből egy, a számítógép által értelmezhető és végrehajtható program.

Ennek megvalósításához szükségünk lesz egy \textbf{fordítóprogram}ra (angolul \textit{compiler}re). A fordítóprogram nem más, mint egy olyan eszköz, ami szöveges bemenetet fogad el (fájl, parancsszori bemenet, stb.), ellenőrzi azt, majd

\begin{itemize}
	\item ha helyes a szövegünk, létrehozza a futtaható programot,
	\item ha nem, hibát jelez (esetleg megmutatja, \textit{mi a hiba} és az \textit{hol található}).
\end{itemize}

Azt a nyelvet, amit a számítógép beszél, \textbf{gépi kód}nak (\textit{machine code}) nevezzük. Ez egy gépközeli nyelv (numerikus utasításkódok, regiszterek, memóriahivatkozások, stb.), mely erősen platformfüggő, cserébe jól optimalizált.

Ezen tulajdonságai kényelmetlenné teszik a programírást a \textbf{magas(abb) szintű nyelvek}kel ellentétben (\textit{high(er)-level languages}), melyekben könnyebb programozni, a benne megírt kód közelebb a megoldandó problémához, emellett platform-független.

\begin{figure}[h!]
	\begin{minipage}{0.5\linewidth}
		\begin{verbatim}
			int sum = 0;
			
			for (int i = 0; i < len; ++i)
			{
			    sum += t[i];
			}
		\end{verbatim}
	\end{minipage}
	\begin{minipage}{0.5\linewidth}
		\begin{verbatim}
			B9 00 00 00 00
			B8 00 00 00 00
			81 F9 0A 00 00 00
			7D 06
			03 04 8B
			41
			EB F2
		\end{verbatim}
	\end{minipage}
	
	\caption{Magas szintű nyelv és gépi kódja}
\end{figure}

Megelőlegezzük, hogy a magas szintű nyelvek és a gépi kód között helyezkedik el az \textbf{Assembly}, ami egy \textit{ember számára olvashatóbb} változatát nyújtja a \textit{gépi kód}nak. Kriptikus hexadecimális számok helyett rendkívül egyszerű műveletek, regiszterek, címkék, ugróutasítások állnak a programozó rendelkezésre. Azt a programot, ami egy Assembly-forráskódból gépi kódot generál, \textbf{assembler}nek nevezzük. Bővebben a róla szóló fejezetben les szó.

\begin{figure}
	\centering
	\begin{tikzpicture}
		% High-level source code
		\node[block, minimum width=5cm, minimum height=1.5cm] (highlevel) at (0,0) {};
		\node[align=center] at (highlevel) {magas-szintű\\forráskód};
		
		% Compiler
		\node[block, minimum width=5cm, minimum height=1cm] (compiler) at (0,-2.5) {};
		\node[align=center] at (compiler) {Assembly kód};
		
		% Assembly source code
		\node[block, minimum width=5cm, minimum height=1cm] (assembly) at (0,-5) {};
		\node[align=center] at (assembly) {gépi kód};
		
		% Arrows
		\draw[->, thick] (highlevel.south) -- (compiler.north) node[midway, right] {compiler};
		\draw[->, thick] (compiler.south) -- (assembly.north) node[midway, right] {assembler};
	\end{tikzpicture}
	\caption{A forráskód állapotának szakaszai}
\end{figure}

\section{Fajtái}

A programozási nyelveket háromféle csoportba oszthatjuk attól függően, milyen stratégiát követ az adott nyelv fordítóprogramja.

\subsection{Fordított programozási nyelvek}

Léteznek az ún. \textbf{fordított programozási nyelvek} (\textit{compiled programming languages}, \ref{compiledlang}. ábra), melyeknek fordítóprogramja generál egy, a gépen közvetlenül futtatható állományt. A fordítás folyamata lassabb, azonban a létrejött program végrehajtása gyors. Ha hiba merül fel, két időszakaszban jelentkezhetnek ezek.
\begin{itemize}
	\item \textbf{Fordítási idő}ben történik (\textit{compile time}), ha a compiler veszi észre, jelez róla vissza -- ekkor megszakítja a fordítást. Az ilyen hibát fordítási idejű hibának hívjuk.
	\item \textbf{Futási idő}ben történik (\textit{run time} vagy \textit{runtime}), ha a fordítás sikeres volt, létrejött a futtatható gépi kód, de működés közben elszáll. Ezek a futási idejű hibák.
\end{itemize}

Az ilyen nyelvekben a forráskód alaposabb ellenőrzése erősen javasolt.

További előnye a fordított nyelveknek, hogy a fordító képes a \textbf{tárgykódot optimalizálni}, akár az adott platformra specifikusan. Hátránya sajnos ebben is rejlik, hiszen \textbf{minden platformra külön-külön le kell fordítanunk}.

\begin{figure}[h!]
	\centering
	\begin{tikzpicture}
		\node[block, minimum width=2.5cm, minimum height=1cm] (src) at (-0.5,0) {};
		\node[align=center] at (src) {forráskódok};
		
		\node[block, minimum width=3.5cm, minimum height=1cm] (compiler) at (4,0) {};
		\node[align=center] at (compiler) {\textbf{fordítóprogram}};
		
		\node[block, minimum width=2.5cm, minimum height=2cm] (input) at (10,3) {};
		\node[align=center] at (input) {bemenet, \\ környezet \\ eseményei};
		
		\node[block, minimum width=2.5cm, minimum height=1.5cm] (obj) at (10,0) {};
		\node[align=center] at (obj) {tárgykód \\ (object code)};
		
		\node[block, minimum width=4cm, minimum height=2cm] (output) at (10,-3) {};
		\node[align=center] at (output) {kimenet, \\ program által \\ kiváltott események};
		
		\node[selection, minimum width=4cm, minimum height=3cm] (compiletime) at (4,0) {};
		\node at (4, -1.25) {\textit{fordítási idő}};
		
		\node[selection, minimum width=5cm, minimum height=10cm] (runtime) at (10,0) {};
		\node at (10,-4.75) {\textit{futási idő}};
		
		\draw[->, thick] (src.east)      -- (compiler.west) node[midway, right] {};
		\draw[->, thick] (compiler.east) -- (obj.west)      node[midway, right] {};
		\draw[->, thick] (input.south)   -- (obj.north)     node[midway, right] {};
		\draw[->, thick] (obj.south)     -- (output.north)  node[midway, right] {};
	\end{tikzpicture}
	\caption{Fordítás és végrehajtás}
	\label{compiledlang}
\end{figure}

Tipikusan ilyen nyelvek: C, C++, Haskell, Ada, \dots

\subsection{Értelmezett programozási nyelvek}

A másik nagy csoportot képezik az \textbf{értelmezett} vagy \textbf{interpretált programozási nyelvek} (\ref{interpretedlang}. ábra). Ez némi rugalmasságot enged meg a fordított nyelvekkel szemben. Itt a fordító sorrol sorra hajtja végre az utasításokat és ott áll meg, ahol a hiba jelentkezik -- azaz, \textbf{csak futási idő van}. További következménye, hogy jellemzően \textbf{jelentősen lassabb a végrehajtás}. Cserébe \textbf{minden platformon azonnal futtatható}, ahol az interpreter rendelkezésre áll.

\begin{figure}[h!]
	\centering
	\begin{tikzpicture}
		\node[block, minimum width=2.5cm, minimum height=1cm] (src) at (-0.5,0) {};
		\node[align=center] at (src) {forráskódok};
		
		%\node[block, minimum width=3.5cm, minimum height=1cm] (compiler) at (4,0) {};
		%\node[align=center] at (compiler) {\textbf{fordítóprogram}};
		
		\node[block, minimum width=2.5cm, minimum height=2cm] (input) at (10,3) {};
		\node[align=center] at (input) {bemenet, \\ környezet \\ eseményei};
		
		\node[block, minimum width=2.5cm, minimum height=1.5cm] (obj) at (10,0) {};
		\node[align=center] at (obj) {\textbf{értelmező} \\ (interpreter)};
		
		\node[block, minimum width=4cm, minimum height=2cm] (output) at (10,-3) {};
		\node[align=center] at (output) {kimenet, \\ program által \\ kiváltott események};
		
		%\node[selection, minimum width=4cm, minimum height=3cm] (compiletime) at (4,0) {};
		%\node at (4, -1.25) {\textit{fordítási idő}};
		
		\node[selection, minimum width=5cm, minimum height=10cm] (runtime) at (10,0) {};
		\node at (10,-4.75) {\textit{futási idő}};
		
		\draw[->, thick] (src.east)      -- (obj.west)      node[midway, right] {};
		%\draw[->, thick] (compiler.east) -- (obj.west)      node[midway, right] {};
		\draw[->, thick] (input.south)   -- (obj.north)     node[midway, right] {};
		\draw[->, thick] (obj.south)     -- (output.north)  node[midway, right] {};
	\end{tikzpicture}
	\caption{Értelmezés}
	\label{interpretedlang}
\end{figure}

Tipikusan ilyen nyelvek: Python, Perl, PHP, JavaScript, \dots

\subsection{Fordítás végrehajtás közben}

Létezik a két stratégiának az ötvözete, a \textbf{fordítás végrehajtás közben} (angolul \textit{just-in-time (JIT) compilation}).

Hasonlóan a fordított programozási nyelvekhez, a fordítási és futási idő elkülönül. A fordítóprogram gépi kód helyett \textbf{bájtkód}ra (\textit{bytecode}) fordítja le a forráskódot. Ezután a \textbf{virtuális gép} (\textit{virtual machine}) végrehajtja a bájtkód utasításait. Azonban felmerülnek bizonyos problémák.
\begin{enumerate}
	\item Ha a virtuális gép \textit{futási időben értelmezi} a bájtkódot, az ugyanolyan lassan történne, mint egy hagyományos értelmezett nyelv esetében.
	\item Ha \textit{végrehajtás előtt fordítjuk le} teljesen a bájtkódot gépi kódra, az túl nagy kezdeti lassulást eredményezne.
\end{enumerate}

Éppen ezért a következő stratégiát követi a virtuális gép.
\begin{enumerate}
	\item Kezdetben értelmezi, interpretálja a bájtkódot.
	\item Futási időben statisztikákat gyűjt a leggyakrabban lefutó kódrészletekről. Ezeket ``\textit{hot spot}''-oknak nevezzük.
	\item Ezeket lefordítja gépi kódra.
	\item Így a következő alkalommal a lefordított kódrészlet fut az értelmezés helyett.
\end{enumerate}

További előnye, hogy a JIT fordító futási időben gyűjtött információkat is figyelembe vehet a kódoptimalizálásnál. Ilyenekhez a klasszikus fordítóprogram nem fér hozzá!

Ugyanakkor, a bájtkódok végrehajtása jellemzően még így is lassabb a gépi kódhoz képest, de ez sepciális alkalmazási
területeket leszámítva nem baj.

Tipikusan ilyen nyelvek: C\#, Java.

\begin{figure}[h!]
	\centering
	\begin{tikzpicture}
		\node[block, minimum width=2.5cm, minimum height=1cm] (src) at (-2,0) {};
		\node[align=center] at (src) {forráskódok};
		
		\node[block, minimum width=3.5cm, minimum height=1cm] (compiler) at (2,0) {};
		\node[align=center] at (compiler) {\textbf{fordítóprogram}};
		
		\node[block, minimum width=2cm, minimum height=1.5cm] (bytecode) at (5.75,0) {};
		\node[align=center] at (bytecode) {bájtkód \\ (bytecode)};
		
		\node[block, minimum width=2.5cm, minimum height=2cm] (input) at (10,2.5) {};
		\node[align=center] at (input) {bemenet, \\ környezet \\ eseményei};
		
		\node[block, minimum width=3.5cm, minimum height=1.5cm] (obj) at (10,0) {};
		\node[align=center] at (obj) {\textbf{virtuális gép} \\ (virtual machine)};
		
		\node[block, minimum width=4cm, minimum height=2cm] (output) at (10,-2.5) {};
		\node[align=center] at (output) {kimenet, \\ program által \\ kiváltott események};
		
		\node[selection, minimum width=4cm, minimum height=3cm] (compiletime) at (2,0) {};
		\node at (2, -1.25) {\textit{fordítási idő}};
		
		\node[selection, minimum width=5cm, minimum height=8cm] (runtime) at (10,0) {};
		\node[rotate=90] at (12.25,0) {\textit{futási idő}};
		
		\draw[->, thick] (src.east)      -- (compiler.west) node[midway, right] {};
		\draw[->, thick] (compiler.east) -- (bytecode.west) node[midway, right] {};
		\draw[->, thick] (bytecode.east) -- (obj.west)      node[midway, right] {};
		\draw[->, thick] (input.south)   -- (obj.north)     node[midway, right] {};
		\draw[->, thick] (obj.south)     -- (output.north)  node[midway, right] {};
	\end{tikzpicture}
	\caption{JIT-fordítás}
\end{figure}

\begin{center}
	\begin{tabular}{c|cc}
		%\hline
		\textbf{Nyelv} & \textbf{Bájtkód} & \textbf{Virtuális gép} \\
		\hline
		C\# & Common Intermediate Language (CLI) & Common Language Runtime (CLR) \\
		%\hline
		Java & Java bytecode & Java Virtual Machine (JVM) \\
		%\hline
	\end{tabular}
\end{center}

\section{Fejlődése}

%1957-ben jelent meg a legelső compiler, ami a Fortran nyelvhez készült, 18 emberévnyi munka után.

\begin{itemize}
	\item 1957: Első \textbf{Fortran compiler} -- 18 emberévnyi munka
	\item Azóta fejlődött a \textbf{formális nyelvek és automaták elmélete}.
	\item Ma: A fordítóprogramok létrehozásának egy része \textbf{automatizálható} elemzőgenerátorokkal.
	\begin{itemize}
		\item A programszöveg elemi egységekre (tokenekre) bontása
		\item A programszöveg formai helyességének vizsgálata
	\end{itemize}
	\item A további ellenőrzések és a kódgenerálás nem automatizálható, de az implemetációt \textbf{keretrendszerek} segíthetik.
	\item A \textbf{kódoptimalizálás} (és a hozzá szükséges elemzések) komoly kihívás.
\end{itemize}

\section{Logikai felépítése}

Két fázisból áll a fordítás: az \textbf{analízis}ből és \textbf{szintézis}ből. Az egyes részeket és alrészeket önálló fejezetek is részletezik, itt csak egy rövid áttekintést nyújtunk.

A vizuális összefoglaló a \ref{compilation}. ábrán található.

\subsection{Analízis}

Az analízis \textit{előfeldolgozás}t hajt végre a bemeneti forráskódon. Ellenőrzi, hogy lexikálisan, szintaktikusan, illetve szemantikusan helyes-e a kódunk. Ha ez nincs így, a megfelelő helyen hibát dob.

\subsubsection{Lexikális elemzés}

\begin{itemize}
	\item \textit{Feladat}: A forrásszöveg elemi egységekre, ún. \textbf{token}ekre bontása. Idegen szóval ez a \textbf{tokenizáció}.
	\item \textit{Bemenet}: karaktersorozat
	\item \textit{Kimenet}: tokenek sorozata + lexikális hibák
	\item \textit{Eszközök}: reguláris kifejezések, véges determinisztikus automaták
\end{itemize}

\begin{figure}[h!]
	\centering
	\begin{tikzpicture}
		\node[block, minimum width=3cm, minimum height=1cm] (input) at (0,0) {};
		\node at (input) {\texttt{x = x + 1;}};
		
		\node[block, minimum width=11cm, minimum height=1cm] (output) at (0, -2) {};
		\node at (output) {
			$\langle$
			változó\textsubscript{\texttt{x}}, 
			operátor\textsubscript{\texttt{=}}, 
			változó\textsubscript{\texttt{x}}, 
			operátor\textsubscript{\texttt{+}}, 
			literál\textsubscript{\texttt{1}}, 
			utasításvég
			$\rangle$
		};
	
	\draw[->, thick] (input.south) -- (output.north);
	\end{tikzpicture}
	\caption{Helyes lexikális elemzés eredménye}
\end{figure}


\subsubsection{Szintaktikus elemzés}

\begin{itemize}
	\item \textit{Feladat}: A forrásszöveg szerkezetének felderítése, formai ellenőrzése.
	\item \textit{Bemenet}: tokenek sorozata
	\item \textit{Kimenet}: szintaxisfa + szintaktikus hibák
	\item \textit{Eszközök}: környezetfüggetlen nyelvtanok, veremautomaták
\end{itemize}

\begin{figure}[h!]
	\centering
	\begin{tikzpicture}
		\node[block, minimum width=11cm, minimum height=1cm] (input) at (0, 0) {};
		\node[align=center] at (input) {
			$\langle$
			változó\textsubscript{\texttt{x}}, 
			operátor\textsubscript{\texttt{=}}, 
			változó\textsubscript{\texttt{x}}, 
			operátor\textsubscript{\texttt{+}}, 
			literál\textsubscript{\texttt{1}}, 
			utasításvég
			$\rangle$
		};
	
		\node[block, minimum width=14cm, minimum height=6cm] (output) at (0,-4) {};
		\node at (output) {
			\begin{tikzpicture}[sibling distance=1em]
				\Tree
				[.{Utasítás}
					[.{Kifejezés}
						[.{Kifejezés}
							{változó\textsubscript{\texttt{x}}}
						]
						{operátor\textsubscript{\texttt{=}}}
						[.{Kifejezés}
							[.{Kifejezés}
								{változó\textsubscript{\texttt{x}}}
							]
							{operátor\textsubscript{\texttt{+}}}
							[.{Kifejezés}
								{literál\textsubscript{\texttt{1}}}
							]
						]
					]
					{utasításvég}
				]
			\end{tikzpicture}
		};
		
		\draw[->, thick] (input.south) -- (output.north);
	\end{tikzpicture}
	\caption{Helyes szintaktikus elemzés eredménye}
\end{figure}

Az alábbi környezetfüggetlen nyelv határozza meg a szintaxist. Ezen nyelvnek a terminális szimbólumai a tokenek (lexikális elemek).
\begin{flalign*}
	\text{<Utasítás>} &::= \text{<Kifejezés> utasításvég}
	\\
	\text{<Kifejezés>} &::= \text{változó | literál | <Kifejezés> operátor <Kifejezés>}
\end{flalign*}

\subsubsection{Szemantikus elemzés}

\begin{itemize}
	\item \textit{Feladat}: A statikus szemantika (pl. változók deklaráltsága, típushelyesség stb.) ellenőrzése
	\item \textit{Bemenet}: szintaxisfa
	\item \textit{Kimenet}: \textbf{szintaxisfa attribútumokkal}, \textbf{szimbólumtábla} + szemantikus hibák
	\item \textit{Eszközök}: \textbf{attribútumnyelvtanok}
\end{itemize}

\begin{figure}[h!]
	\centering
	\begin{tabular}{|l|l|}
		\hline
		Név ~~~~~ & Típus ~~~~~ \\
		\hline
		\texttt{x} & \texttt{int} \\
		\hline
	\end{tabular}
	\caption{Szimbólumtáblázat. Általában több információt is tartalmaz.}
\end{figure}

\begin{figure}[h!]
	\centering
	\begin{tikzpicture}[sibling distance=1em]
		\Tree
		[.{Utasítás}
		[.{Kifejezés\textsuperscript{\texttt{void}}}
		[.{Kifejezés\textsuperscript{\texttt{int}}}
		{változó\textsubscript{\texttt{x}}}
		]
		{operátor\textsubscript{\texttt{=}}}
		[.{Kifejezés\textsuperscript{\texttt{int}}}
		[.{Kifejezés\textsuperscript{\texttt{int}}}
		{változó\textsubscript{\texttt{x}}}
		]
		{operátor\textsubscript{\texttt{+}}}
		[.{Kifejezés\textsuperscript{\texttt{int}}}
		{literál\textsubscript{\texttt{1}}}
		]
		]
		]
		{utasításvég}
		]
	\end{tikzpicture}
	\caption{Szintaxisfa attribútumokkal}
\end{figure}

\newpage

\subsection{Szintézis}

\subsubsection{Kódgenerálás}

\begin{itemize}
	\item \textit{Feladat}: Alacsonyabb szintű belső reprezentációkra, végül \textbf{tárgykód}dá alakítja a programot
	\item \textit{Bemenet}: szintaxisfa attribútumokkal, szimbólumtábla
	\item \textit{Kimenet (az utolsó menetben)}: \textbf{tárgykód}
	\item \textit{Eszközök}: \textbf{kódgenerálási sémák}
\end{itemize}

Közvetlenül gépi kódot csak nagyon indokolt esetben érdemes generálni. Helyette Assembly kód (pl. valamely platform Assembly nyelve vagy LLVM) generálható, amit assemblerekkel fordítunk tovább.

Megemlíthetjük az ún. \textbf{transzláció}t is. Ez magas szintű nyelvek közti fordítást jelent. Ez lehet végcél (pl. projektek portolása esetén egyik nyelvről a másikra), ugyanakkor elterjedt nyelvekre való fordítás esetén használhatjuk azok fordítóit a gépi kód / bájtkód előállításához.

\subsubsection{Optimalizáció}

\begin{itemize}
	\item \textit{Feladat}: Kód átalakítása hatékonyságnövelés céljából (pl. sebességnövelés, memóriaigény csökkentés)
	\item \textit{Bemenet}: belső reprezentáció / tárgykód
	\item \textit{Kimenet (az utolsó menetben)}: belső reprezentáció / tárgykód
	\item \textit{Eszközök}: \textbf{Statikus elemzés}, \textbf{transzformációs keretrendszerek}
\end{itemize}

Egyes compilerek több lépésben is optimalizálhatják a kódot. 

Ahogy megtárgyaltuk, a szintézis fázisa úgy kezdődik, hogy rendelkezésünkre áll a szintaxisfa. Ezen ún. \textbf{magas szintű optimalizáció}t hajtanak végre, így kapunk egy optimalizált szintaxisfát\footnote{El tudjuk képzelni, hogy milyen matematikai vonatkozásai lehetnek ennek: egy fagráfot kevesebb úttal vagy csúccsal ``írjunk fel'' úgy, hogy az ezen gráf által felírt ``program'' ekvivalens maradjon az eredetivel. (A megfogalmazás természetesen matematikailag pontatlan, csupán a szemléltetés céljából raktam ide.)}. Ez alapján megtörténik a kódgenerálás, létrejön az Assembly kód. Végül ezt az Assembly kódot optimalizáljuk (\textbf{alacsony szintű optimalizálás}).

\begin{figure}
	\centering
	\begin{tikzpicture}
		\node[block, minimum width=3cm, minimum height=1cm] (src) at (0,1) {};
		\node[align=center] at (src) {FORRÁSKÓD};
		
		\node[selection, minimum width=5cm, minimum height=7cm] (analysis) at (-3, -4) {};
		
		\node[block, minimum width=4.5cm, minimum height=1.25cm] (lexer) at (-3,-2) {};
		\node[align=center] at (lexer) {\textbf{lexikális elemző} \\ (scanner / lexer)};
		
		\node[align=center] (lexererror) at (3, -2) {\textit{\underline{lexikális hibák} $\lightning$}};
		
		\node[block, minimum width=4.5cm, minimum height=1.25cm] (syntax) at (-3,-4) {};
		\node[align=center] at (syntax) {\textbf{szintaktikus elemző} \\ (syntax checker)};
		
		\node[align=center] (syntaxerror) at (3, -4) {\textit{\underline{szintaktikus hibák} $\lightning$}};
		
		\node[block, minimum width=4.5cm, minimum height=1.25cm] (semantics) at (-3,-6) {};
		\node[align=center] at (semantics) {\textbf{szemantikus elemző} \\ (semantic analyser)};
		
		\node[align=center] (semanticerror) at (3, -6) {\textit{\underline{szemantikus hibák} $\lightning$}};
		
		\node[block, minimum width=4.5cm, minimum height=1.25cm] (codegen) at (0,-10) {};
		\node[align=center] at (codegen) {\textbf{kódgeneráló} \\ (code generator)};
		
		\node[block, minimum width=4.5cm, minimum height=1.25cm] (optimiser) at (0,-12) {};
		\node[align=center] at (optimiser) {\textbf{optimalizáló} \\ (optimiser)};
		
		\node[selection, minimum width=5cm, minimum height=5cm] (synthesis) at (0, -11) {};
		
		\node[align=center] at (-3,-7.25) {\textit{analízis}};
		\node[align=center] at (0, -13.25) {\textit{szintézis}};
		
		\node[block, minimum width=2.5cm, minimum height=1cm] (obj) at (0,-16) {};
		\node[align=center] at (obj) {TÁRGYKÓD};
		
		\draw[->, thick] (src.south)      -- (analysis.north) node[midway, right] {};
		
		\draw[->, thick] (lexer.south)    -- (syntax.north)    node[midway, right] {};
		\draw[->, thick] (syntax.south)   -- (semantics.north) node[midway, right] {};
		
		\draw[->, thick, dashed] (lexer.east)  -- (lexererror.west) node[midway, right] {};
		\draw[->, thick, dashed] (syntax.east)  -- (syntaxerror.west) node[midway, right] {};
		\draw[->, thick, dashed] (semantics.east)  -- (semanticerror.west) node[midway, right] {};
		
		\draw[<->, thick] (codegen) -- (optimiser);
		
		\draw[->, thick] (analysis.south) -- (synthesis.north) {};
		
		\draw[->, thick] (synthesis.south) -- (obj.north) {};
	\end{tikzpicture}
	\caption{A fordítóprogramok logikai felépítése}
	\label{compilation}
\end{figure}

\newpage

\section{Szerkesztés és végrehajtás}

Általában amikor programot írunk, modularizálva írjuk meg azt, azaz több, kisebb részekre bontjuk fel -- legtöbbször \textbf{könyvtárak}, \textbf{csomagok} formájában. Ezen összetevők gyakran hivatkoznak egymásra, emiatt elengedhetetlen, hogy el is érjék egymást. Ezt oldja meg a(z) \textbf{(össze)szerkesztés} vagy \textbf{linkelés}.

A mai rendszereken kétféle stratégia létezik a könyvtárak összeszerkesztéséhez.

\begin{enumerate}
	\item \underline{Statikus szerkesztés}
	
	Nagy vonalakban azt jelenti, hogy mindazon \textbf{könyvtárakat, csomagokat}, melyeket felhasználunk a programunkban, \textbf{``beleégetjük'' a gépi kódba}. Tipikusan ez történik, amikor C-ben include-oljuk az \texttt{stdio.h} könyvtárat. Hiába csak a \texttt{printf} függvényt használjuk fel, minden más is bekerül a binárisba.
	
	Ez előnyös lehet, mivel csökkenti a külső függőségeket (akár használhatjuk a programunkat olyan rendszeren, amin nincs telepítve a \texttt{glibc}). Hátránya, hogy jelentősen megnövelheti a futtatható fájl méretét.
	
	A statikus könyvtárak tipikus kiterjesztései: \texttt{.a}, \texttt{.lib}.
	
	\item \underline{Dinamikus szerkesztés}
	
	Futási időben éri el a hivatkozott függvényeket, osztályokat, stb. Előnye, hogy kisebb lesz a futtatandó fájl mérete. Hátránya pedig, hogy meg kell győződnünk futtatás előtt, hogy telepítve vannak-e a szükséges \textbf{függőségek}.
	
	A dinamikus könyvtárak tipikus kiterjesztései: \texttt{.so} (\textit{shared object}), \texttt{.dll} (\textit{dynamically linked library}).
	
	Ha visszaemlékszünk az \textit{Objektumelvű programozás} c. tárgyból tanultakra, a gyakorlatokon használtuk a \texttt{TextFileReader.dll} könyvtárat, ami (a mostani tudásunkkal összevetve) egy dinamikusan linkelt könyvtár.
\end{enumerate}

A programunk futtatása, végrehajtása esetén a \textbf{teljes futtatható állományt betöltjük a fájlrendszerből}. Ha dinamikus könyvtárakat is használunk, ezek is betöltésre kerülnek.