%\chapter{Formális nyelvek}

%Az előadások ezen részét \textsc{Nagy Sára} tartotta.

%\section{Mi az, hogy formális nyelv?}

\chapter{Szavak és nyelvek}

\section{Alapvető fogalmak}

\begin{tcolorbox}
	\begin{definition}[Ábécé]
		Egy $\Sigma$ véges és nemüres halmazt \textbf{ábécé}nek hívunk.
		%\[ \Sigma \neq \emptyset ~ \land ~ |\Sigma| < \infty. \] 
		Ennek elemeit \textbf{betű}knek hívjuk.
	\end{definition}
\end{tcolorbox}

Az ábécé jele a szakirodalomban változhat -- például az előadáson $V$-vel jelöltük, azonban \textit{Algoritmusok és adatszerkezetek II}.-ből $\Sigma$ volt a jele. A jegyzet ezt az utóbbit fogja használni.

\begin{tcolorbox}
	\begin{definition}[Szó és hossza]
		Az $u \in \Sigma^*$ véges sorozatot egy \textbf{sztring}nek vagy \textbf{szó}nak nevezzük, melynek hosszát az $\ell : \Sigma^* \to \mathbb{N}$ függvény jelöli úgy, hogy \[ \forall u \in \Sigma^* : 0 \leq \ell(u) < \infty. \] Speciális esete az \textbf{üres szó}, melynek jele $\boxed{\emptyword}$ és $\ell(\emptyword) = 0$.
	\end{definition}
\end{tcolorbox}

A \textit{sztring} és \textit{szó} elnevezés felcserélhető, ám jellemzően szónak hívunk egy véges betűsorozatot, ha tudjuk róla, hogy az egy nyelvnek egy szava.

A $\Sigma^*$ jelöli azon véges sorozatok halmazát, melyeket a $\Sigma$ ábécé betűiből képeztünk. Ennek eleme az üres sorozat vagy üres szó is, azaz $\boxed{\emptyword \in \Sigma^*}$. Megállapodunk abban, hogy \[ \boxed{\Sigma^+ := \Sigma^* \setminus \{ \emptyword \}} . \]

A \textbf{legszűkebb ábécé}t egy szóra nézve az alábbi módon jelöljük: $\Sigma(u) \subseteq \Sigma$.

Egyes szerzők az abszolútérték jelet használják a szó hosszának jelölésére, azaz $|u| = \ell(u)$. A jegyzet az $\ell$ betűvel fogja jelölni, ugyanis ez kevésbé félreérthető.

Lekérdezhetjük, hogy egy adott szó mennyit tartalmaz egy adott betűből. Például ha $ \Sigma := \{ \texttt{a}, \texttt{b} \} $, akkor \[ \ell_\texttt{a}(u) ~~~ (u \in \Sigma^*)\] azt jelöli, hány darab \texttt{a} betű található az $u$ szóban.
\begin{tcolorbox}
	\begin{definition}[Nyelv]
		A $L \subseteq \Sigma^*$ halmazt \textbf{nyelv}nek nevezzük (azaz a nyelv egy halmaz, ami szavakat tartalmaz).
	\end{definition}
\end{tcolorbox}

%\newpage

Speciális nyelvek:
\begin{itemize}
	\item üres nyelv: $\boxed{\emptyset}$ vagy $\boxed{L_\emptyset}$
	\item üres szót tartalmazó nyelv: $\boxed{\{\emptyword\}}$ vagy $\boxed{L_\emptyword}$
\end{itemize}

Annak ellenére, hogy a halmaz rendezetlenül tárolja az elemeit, hagyományosan \textbf{lexikografikus sorrend}ben szoktuk felsorolni a nyelv szavait. Ez ábécé szerinti elsődleges és hossz szerinti másodlagos rendezést jelent.

\begin{tcolorbox}
	\begin{definition}[Nyelvcsalád]
		Legyenek $L_1, L_2, \dots, L_k \subseteq \Sigma^*$ $(k \in \mathbb{N}^+)$ nyelvek egy ábécé felett. Ekkor a $\mathcal{L} := \{ L_1, L_2, \dots, L_k \}$ halmazt \textbf{nyelvcsalád}nak vagy \textbf{nyelvosztály}nak hívjuk.
	\end{definition}
\end{tcolorbox}

\section{Műveletek}

\subsection{Műveletek szavak felett}

\begin{tcolorbox}
	\begin{definition}[Konkatenáció]
		Legyen $u := u_1u_2\dots u_n$ és $v := v_1v_2\dots v_m$ két szó $\Sigma^*$ felett $(u, v \in \Sigma^*)$. Ekkor \[ uv := u_1u_2\dots u_n v_1v_2\dots v_m \] az $u$ és $v$ \textbf{konkatenáció}ja $(uv \in \Sigma^*)$. Jele általában nincs (néha ponttal $(\cdot)$ jelezzük).
	\end{definition}
\end{tcolorbox}

A \textbf{konkatenáció tulajdonságai} -- $\forall u, v, w \in \Sigma^*:$
\begin{enumerate}
	\item asszociatív: $u(vw) = (uv)w$,
	\item nem kommutatív: $uv \neq vu$,
	\item $\Sigma^*$-ra zárt művelet (nem vezet ki a halmazból),
	\item egységeleme az üres szó ($\emptyword$): $\emptyword u = u \emptyword = u$,
	\item $(\Sigma^*, \cdot, \emptyword )$ egy egységelemes félcsoportot alkot.
\end{enumerate}

\begin{tcolorbox}
	\begin{definition}[Hatványozás]
		Legyen $u \in \Sigma^*$ és $n \in \mathbb{N}$.
		\[ u^n := \begin{cases}
			\emptyword ~~~~~~~~~~~ (n = 0) \\
			u ~~~~~~~~~~~ (n = 1) \\
			u^{n-1}u ~~~~~ (n > 1).
		\end{cases} \]
	\end{definition}
\end{tcolorbox}

A fenti definíció \textit{balrekurzív}, de ugyanúgy működne, ha jobbrekurzívan definiálnánk.

A \textit{konkatenáció}t és a \textit{hatványozás}t \textbf{reguláris műveletek}nek nevezzük.

\begin{tcolorbox}
	\begin{definition}[Megfordítás]
		Legyen $ u_1 u_2 \dots u_n =: u \in \Sigma^*$. Ekkor \[ u^R := u^{-1} := u_n u_{n-1} \dots u_2 u_1 \] Jele: $\boxed{u^{-1}}$ vagy $\boxed{u^R}$.
	\end{definition}
\end{tcolorbox}

Ismertetünk további, szavakkal kapcsolatos alapfogalmakat. $\forall u, v \in \Sigma^*,$

\begin{enumerate}
	\item \textbf{Részszó}: $\exists w_1, w_2 \in \Sigma^* : u = w_1 v w_2$.
	\item \textbf{Prefix}: $v \sqsubseteq u \Longleftrightarrow \exists w \in \Sigma^* : u = vw$.
	\item \textbf{Szuffix}: $u \sqsupseteq v \Longleftrightarrow \exists w \in \Sigma^* : u = wv$.
	\item \textbf{Valós prefix ($\sqsubset$), valós szuffix ($\sqsupset$)}: a megfelelő definíció, továbbá $v \neq \emptyword \land v \neq u$.
\end{enumerate}

A jelöléseket az \textit{Algoritmusok és adatszerkezetek II}. jegyzetből kölcsönöztem.

\subsection{Műveletek nyelvek felett}

Az eddig megismert, szavakon értelmezett műveleteket kiterjesztjük a nyelvek szintjére, valamint a már jól ismert halmazműveleteket is megvizsgáljuk, hogyan viselkednek a nyelvek felett.

\begin{tcolorbox}
	\begin{definition}[Unió]
		Legyen $L_1, L_2 \subseteq \Sigma^*$. Ekkor \[ L_1 \cup L_2 := \{ u \in \Sigma^* \mid u \in L_1 \lor u \in L_2 \}. \]
	\end{definition}
\end{tcolorbox}

Tulajdonságai:

\begin{enumerate}
	\item kommutatív: $L_1 \cup L_2 = L_2 \cup L_1$
	\item asszociatív: $L_1 \cup (L_2 \cup L_3) = (L_1 \cup L_2) \cup L_3$
	\item egységeleme a(z) üres nyelv ($L_\emptyset$): $L \cup L_\emptyset = L_\emptyset \cup L = L$
\end{enumerate}

\begin{tcolorbox}
	\begin{definition}[Metszet]
		Legyen $L_1, L_2 \subseteq \Sigma^*$. Ekkor \[ L_1 \cap L_2 := \{ u \in \Sigma^* \mid u \in L_1 \land u \in L_2 \}. \]
	\end{definition}
\end{tcolorbox}

\begin{tcolorbox}
	\begin{definition}[Komplementer]
		Legyen $L \subseteq \Sigma^*$. Ekkor \[ \bar{L} := \Sigma^* \setminus L. \]
	\end{definition}
\end{tcolorbox}

Tulajdonságai:

\begin{enumerate}
	\item $L \cup \bar{L} = \Sigma^*$
	\item $L \cap \bar{L} = L_\emptyset$
\end{enumerate}

\begin{tcolorbox}
	\begin{definition}[Konkatenáció]
		Legyen $L_1, L_2 \subseteq \Sigma^*$. Ekkor \[ L_1L_2 := \{ uv \mid u \in L_1 \land v \in L_2 \}. \]
	\end{definition}
\end{tcolorbox}

Tulajdonságai:

\begin{enumerate}
	\item a nyelvek felett is asszociatív, de nem kommutatív (ahogyan a szavak esetében)
	\item egységeleme az üres nyelvet tartalmazó nyelv ($L_\emptyword$): $LL_\emptyword = L_\emptyword L = L$
	\item a nyelvek halmaza a konkatenációra nézve egység elemes félcsoport alkot
	\item kétoldali disztributivitás áll fenn az unióval:
	\begin{flalign*}
		L(L_1 \cup L_2) = LL_1 \cup LL_2 \\
		(L_1 \cup L_2)L = L_1 L \cup L_2 L
	\end{flalign*}
	\item vigyázat: a metszettel nem áll fenn a disztributivitás:
\end{enumerate}

\begin{tcolorbox}
	\begin{definition}[Nyelv hatványa]
		Legyen $L \subseteq \Sigma^*$ és $n \in \mathbb{N}$.
		\[ L^n := \begin{cases}
			L_\emptyword ~~~~~~~~~~ (n = 0) \\
			L ~~~~~~~~~~~ (n = 1) \\
			L^{n-1}L ~~~~~ (n > 1).
		\end{cases} \]
	\end{definition}
\end{tcolorbox}

Felhívjuk a figyelmet a következő, látszólag hasonló, ám eltérően működő műveletre.

\begin{tcolorbox}
	\begin{definition}[Nyelv megfordítása]
		Legyen $L \subseteq \Sigma^*$ nyelv. Ekkor
		\[ L^{-1} := \left\{ u^{-1} \setdivbar u \in L \right\} \]
		jelöli az $L$ \textbf{nyelv megfordításá}t.
	\end{definition}
\end{tcolorbox}

A nyelv megfordításának jelentése: minden szavát megfordítjuk. Ellenben a \textbf{nyelv hatványra emelése} arról szól, hogy a nyelv szavait összekonkatenáljuk egymással az összes lehetséges módon -- vagyis \textbf{nem szavankénti hatványozást jelent}!

A következő művelet a hatványozást ``emeli egy magasabb szintre''.

\begin{tcolorbox}
	\begin{definition}[Nyelv lezártja, iteráltja]
		Legyen $L \subseteq \Sigma^*$. Ekkor \[ L^* := L^0 \cup L^1 \cup L^2 \cup L^3 \cup \dots = \bigcup_{i \geq 0} L^i . \] Pozitív lezártja: \[ L^+ := L^* \setminus L_\emptyword = \bigcup_{i \geq 1} L^i . \]
	\end{definition}
\end{tcolorbox}

Az alábbi műveleteket nevezzük \textbf{reguláris műveletek}nek: \framebox{unió, konkatenáció, lezárás}.

\iffalse

\section{Feladatok}

\begin{enumerate}
	\item  Legyen $\Sigma := \{ \texttt{a},\texttt{b},\texttt{c} \}$ és legyen $u_1 := \texttt{cca}$, $u_2 := \texttt{aabc}$ egy–egy $\Sigma$ feletti szó. \\
	Soroljuk fel $u_1$ és $u_2$ valódi részszavait, adjuk meg a hosszukat, konkatenáltjukat, tükörképüket, 0-adik, 1., 2., 3. hatványukat!
	\item  Döntsük el, hogy az alábbi nyelvek végesek vagy végtelenek! A végteleneket kezdjük el felsorolni
	lexikografikusan!
	\begin{enumerate}
		\item $L = \emptyset$ (üres nyelv)
		\item $L=\{\emptyword\}$ (üres szót tartalmazó nyelv)
		\item $L=\{\texttt{a}^n\texttt{b}^n \mid n \geq 0\}$
		\item $L=\{\texttt{a}\}^*\{\texttt{b}\}^*$ (ez még nem reg.kif., hanem nyelvi műveletekkel felírt kifejezés)
		\item $L=\{ \texttt{a}^n\texttt{b}^k \mid n \geq 0 \land k \geq 0\}$
		\item $L=\left\{ u \in \{\texttt{a},\texttt{b}\}^* \setdivbar u=u^{-1} \right\}$ (palindrom szavak)
		\item $L= \left\{ uu^{-1} \setdivbar u\in \{\texttt{a},\texttt{b}\}^* \right\}$ (szimmetrikus szavak, ami részhalmaza a palindromáknak)
	\end{enumerate}
	\item Igaz-e a disztributivitás? $(L_1 \cup L_2)L_3 = L_1L_3 \cup L_2L_3$
	\item Igaz-e a disztributivitás? Ha nem, akkor adjon ellenpéldát! \[ (L_1 \cap L_2)L_3 = L_1L_3 \cap L_2L_3 \]
	\item Legyen $L_1 := \{ \texttt{a} \}^*\{ \texttt{ba} \}^*$, $L_2 := \{ \texttt{b}^n\texttt{a} \mid n \geq 0 \}$. Mivel egyenlők az alábbi nyelvek?
	\begin{enumerate}
		\item $L_2 \cap L_1 = ~?$
		\item $L_2 \setminus L_1 = ~?$
		\item $L_2^* \cap L_1^* = ~?$
	\end{enumerate}
	\item Legyen $L_1 := \{ \texttt{a}^n\texttt{b}^m \mid m \geq n \geq 0 \}$ és $L_2 := \{ \texttt{ab} \}^*$. Adja meg az alábbi nyelveket!
	\begin{enumerate}
		\item $L_1 \cap L_2 = ~?$
		\item $L_1 \setminus L_2^* = ~?$
		\item $L_2 \setminus L_1^* = ~?$
	\end{enumerate}
	\item Legyen $L_1 := \{ \texttt{ab}, \texttt{a} \}$ és $L_2 := \{ \texttt{a}^k\texttt{b}^n \mid k \geq 1, n \geq 0 \}$. Mivel egyenlők az alábbi nyelvek?
	\begin{enumerate}
		\item $L_1 \cap L_2 = ~?$
		\item $L_1 \setminus L_2 = ~?$
		\item $L_2 \setminus L_1^* = ~?$
	\end{enumerate}
	\item Azonos vagy nem azonos?
	\begin{enumerate}
		\item $L^* \setminus \{\emptyword\} = L^+$
		\item $L^* = L^* L^*$
		\item $(L_1 \cup L_2)^* = (L_1^* L_2^* )^*$
	\end{enumerate}
	\item Mikor igaz?
	\begin{enumerate}
		\item $\emptyset \cup L = L$
		\item $\{\emptyword\} \cup L = L$
		\item $\emptyset L = L$
		\item $\{ \emptyword \}L = L$
	\end{enumerate}
	%\newpage
	\item Igazak-e az alábbi tartalmazások? $L_1 := \{ \texttt{a}^n\texttt{b}^n \mid n \geq 0 \}$ és $L_2 := \left\{ \texttt{a}^{2n+1}\texttt{b} \mid n \geq 0 \right\}$.
	\begin{enumerate}
		\item $\{ \texttt{a}^n\texttt{b}^n\texttt{a}^n\texttt{b} \mid n \geq 0 \} \subseteq L_1L_2$
		\item $\{ \texttt{a}^n\texttt{b}^n\texttt{a}^{2n+1}\texttt{b} \mid n \geq 0 \} \subseteq L_1L_2$
		\item $\{ (\texttt{a}^n\texttt{b}^n)^n \mid n \geq 0 \} \subseteq L_1^*$
		\item $\{ (\texttt{ab})^n \mid n \geq 0 \} \subseteq L_2^+$
	\end{enumerate}
\end{enumerate}

\fi
