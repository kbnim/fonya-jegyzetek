\chapter{Kódgenerálás}

Elérkeztünk a kurzus azon pontjára, amikor is megkoronázzuk az eddigi munkálatainkat, azaz a fordítóprogramunk működő számítógépes programot képes gyártani!

\section{Kódgenerálás attribútumnyelvtannal}

Kódot generálni általában úgy szoktunk, hogy \textbf{transzlációval} a saját, magas szintű nyelvünket lefordítjuk \textbf{Assemblyre}, majd az assembler legenerálja nekünk a kívánt gépi kódot. Kizárólag indokolt esetben érdemes közvetlenül gépi kódot generálni.

Ott hagytuk abba, hogy a szemantikus elemző ellenőrizte a deklarációkat a szimbólumtáblával és a típusokat az attribútumnyelvtannal. Haladjunk ezen az útvonalon -- pontosabban az alulról felfele, $LR$ elemző megoldásának útján.

A szabályok bal oldalaihoz, mint láttuk, felvettünk attribútumokat, melyek különféle tulajdonságokat határoztak meg. \textit{Miért ne tehetnénk meg ugyanezt magával a kód generálásával}?

\textbf{Vegyük fel} a \textit{szabályok bal oldalához} a \textbf{következő szintetizált attribútumot}: $code : \mathbb{S}$. Ez egy egyszerű sztring lesz, amihez hozzákonkatenáljuk az egyes részeket, ahogy haladunk felfele a szintaxisfában.

A nyelvtan továbbra is \[\prodrule{E}{\texttt{int\_literal} \mid \texttt{identifier} \mid E ~ \texttt{less\_than} ~ E}.\]

A mondatunk: \[ \texttt{10 < x}. \]

Az elemzett szintaxisfa eddigi állapota:

\begin{figure}[h]
	\centering
	\begin{forest}
		for tree={edge={-}, l sep-=10pt, s sep+=10pt}
		[$E$\textsuperscript{$type : \text{bool}$}
		[$E$\textsuperscript{$type : \text{int}$}
		[\texttt{int\_literal}\textsuperscript{$value : 10$}, tier=terminal]
		]
		[\texttt{less\_than}, tier=terminal]
		[$E$\textsuperscript{$type : \text{int}$}
		[\texttt{identifier}\textsuperscript{$name : \texttt{x}$}, tier=terminal]
		]
		]
	\end{forest}
\end{figure}

A korábban definiált, szabályokhoz rendelt \textbf{akciókat módosítsuk} úgy, hogy megadjuk az Assembly kódrészleteket.

\begin{stuki*}[14cm]{$\prodrule{E}{\texttt{int\_literal}}$}
	\stm{E.type := \text{int}}
	\stm{E.code := "\texttt{mov eax ,}" + \texttt{int\_literal}.value +"\texttt{\textbackslash n}"}
\end{stuki*}

\begin{stuki*}[14cm]{$\prodrule{E}{\texttt{identifier}}$}
	%\stm{\text{/* Ellenőrzés: Deklarálva volt-e az azonosító? */}}
	\begin{IF}[20]{5}{\stm{\texttt{identifier}.name \notin symbol\_table}}
		\stm{error(\dots)}
		\ELSE
		\stm{type := symbol\_table.get\_type(\texttt{identifier}.name)}
		\stm{E.type := type}
		\begin{CASE}{2}{2}
			\WHEN{\stm{E.type = \text{int}}}
			\stm[2]{E.code := \\ "\texttt{mov eax, [}" + type + "\texttt{]\textbackslash n}"}
			\WHEN{\stm{\dots}}
			\stm{\text{/* Ha logikai lenne, az} \\ \text{kisebb regiszterben is elférne */}}
		\end{CASE}
	\end{IF}
\end{stuki*}

\begin{stuki*}[14cm]{$\prodrule{E_1}{E_2 ~ \texttt{less\_than} ~ E_3}$}
	%\stm{\text{/* Ellenőrzés: Megfelelő típusúak-e a ``\texttt{<}'' operátor paraméterei? */}}
	\begin{IF}[20]{4}{\stm{E_2.type \neq \text{int} \lor E_3.type \neq \text{int}}}
		\stm{error(\dots)}
		\ELSE
		\stm{E_1.type := \text{bool}}
		\stm[3]{E_1.code :=  E_2.code + "\texttt{push eax\textbackslash n}" + \\ E_3.code + "\texttt{mov ebx, eax\textbackslash n}" + "\texttt{pop eax \NEWLINE}" + \\ "\texttt{cmp eax, ebx\NEWLINE}" + "\texttt{mov al, 0\NEWLINE}" + "\texttt{cmovb al, 1\NEWLINE}"}
	\end{IF}
\end{stuki*}

Ahhoz, hogy megkönnyítsük a munkánkat, bevetejünk ún. kódgenerálási sémákat, melyek hatékonyabbá teszik az Assembly kódok ``megkomponálását''.

\section{Kódgenerálási sémák}

A sémákat nem szabványos struktogramok formájában írom fel.

A számokat 32 bites, előjeles egész számként fogjuk tárolni. Minden szám típusú kifejezés eredményét az \texttt{eax} regiszterbe fogjuk kiszámolni.

\begin{figure}[h]
	\begin{stuki*}[5cm]{$\prodrule{E}{\texttt{int\_literal}}$}
		\stm*{\texttt{mov eax,} \textit{literál értéke}}
	\end{stuki*}
	\caption{Számliterál kifejezés kódgenerálási sémája}
\end{figure}

A logikai értékeket 1 bájton fogjuk tárolni. Adatreprezentáció: $false = 0$ és $true = 1$. Minden logikai típusú kifejezés eredményét az \texttt{al} regiszterbe fogjuk kiszámolni.

\begin{figure}[h]
	\begin{minipage}{0.5\linewidth}
		\begin{stuki*}[5cm]{$\prodrule{E}{\texttt{false}}$}
			\stm*{\texttt{mov al, 0}}
		\end{stuki*}
	\end{minipage}
	\begin{minipage}{0.5\linewidth}
		\begin{stuki*}[5cm]{$\prodrule{E}{\texttt{true}}$}
			\stm*{\texttt{mov al, 1}}
		\end{stuki*}
	\end{minipage}
	\caption{Logikai literál kifejezés kódgenerálási sémája}
\end{figure}

A változók adatait a szimbólumtáblában tároljuk. Mindegyikhez egy egyedi címkét generálunk, amikor betesszük őket a táblába. A változó használatakor kiolvassuk a címkét, és beépítjük a generált kódba.

\begin{figure}[h]
	\begin{minipage}{0.5\linewidth}
		\begin{stuki*}[5cm]{$\prodrule{E}{\texttt{identifier}}$}
			\stm*{\texttt{mov eax,} \texttt{[}\textit{címke}\texttt{]}}
		\end{stuki*}
	\end{minipage}
	\begin{minipage}{0.5\linewidth}
		\begin{stuki*}[5cm]{$\prodrule{E}{\texttt{identifier}}$}
			\stm*{\texttt{mov al,} \texttt{[}\textit{címke}\texttt{]}}
		\end{stuki*}
	\end{minipage}
	\caption{Változó mint kifejezés kódgenerálási sémája}
\end{figure}

Ehhez megtesszük az értelemszerű módosításokat a szimbólumtábla implementációjában is.

\begin{figure}[h]
	\centering
	
	\struct{SymbolTable}
	+ $name : \mathbb{S}$ \\
	\vdots \\
	%+ $kind : Kind$ \\
	%+ $type : Type$ \\
	%+ $declaration : \mathbb{N} \times \mathbb{N}$ ~ \texttt{/* (row, column) */} \\
	%+ $used\_here : \mathbb{N} \times \mathbb{N}\langle\rangle$ \texttt{/* list of coordinates */}\\
	+ $code : \mathbb{S}$ \\
	+ $label : \mathbb{S}$ \\
	\hline
	+ SymbolTable(\dots) \\
	\vdots \\
	+ \underline{next\_label$():\mathbb{S}$} \\ 
	\texttt{/* additional methods may come here */} \\
	\eoStruct
	\caption{A szimbólumtábla egy lehetséges módosítása}
\end{figure}

Beépített operátorok kódgenerálási sémája:

\begin{figure}[h]
	\begin{minipage}{0.5\linewidth}
		\begin{stuki*}[5cm]{$\prodrule{E_1}{E_2 \texttt{ plus\_op } E_3}$}
			\stm*[6]{
				\textit{$E_2$ kódja} \\
				\texttt{push eax} \\
				\textit{$E_3$ kódja} \\
				\texttt{mov ecx, eax} \\
				\texttt{pop eax} \\
				\texttt{add eax, ecx}}
		\end{stuki*}
	\end{minipage}
	\begin{minipage}{0.5\linewidth}
		\begin{stuki*}[5cm]{$\prodrule{E}{\texttt{identifier}}$}
			\stm*[6]{
				\textit{$E_2$ kódja} \\
				\texttt{push eax} \\
				\textit{$E_3$ kódja} \\
				\texttt{mov ecx, eax} \\
				\texttt{pop eax} \\
				\texttt{mul eax, ecx}}
		\end{stuki*}
	\end{minipage}
	\caption{Összeadás és szorzás kódgenerálási sémája}
\end{figure}

\begin{figure}[h]
	\begin{minipage}{0.5\linewidth}
		\begin{stuki*}[5cm]{$\prodrule{E_1}{E_2 \texttt{ plus\_op } E_3}$}
			\stm*[7]{
				\textit{$E_2$ kódja} \\
				\texttt{push eax} \\
				\textit{$E_3$ kódja} \\
				\texttt{mov ecx, eax} \\
				\texttt{pop eax} \\
				\texttt{cmp eax, ecx} \\
				\texttt{mov al, 0} \\
				\texttt{cmovb al, 1}}
		\end{stuki*}
	\end{minipage}
	\begin{minipage}{0.5\linewidth}
		\begin{stuki*}[5cm]{$\prodrule{E}{\texttt{identifier}}$}
			\stm*[6]{
				\textit{$E_2$ kódja} \\
				\texttt{push ax} \\
				\textit{$E_3$ kódja} \\
				\texttt{mov cx, ax} \\
				\texttt{pop ax} \\
				\texttt{and al, cl}}
		\end{stuki*}
	\end{minipage}
	\caption{Kisebb operátor és logikai ``és'' kódgenerálási sémája}
\end{figure}

\pagebreak

Értékadás kódgenerálási sémája egész számra (bal oldal) és logikai értékre (jobb oldal). A változó címkéjét itt is a szimbólumtáblából vesszük.

\begin{figure}[h]
	\begin{minipage}{0.5\linewidth}
		\begin{stuki*}[5cm]{$\prodrule{U}{ \texttt{identifier} ~ \texttt{assign\_op} ~ E}$}
			\stm*[2]{
				\textit{E kódja} \\
				\texttt{mov [\textit{címke}], eax}}
		\end{stuki*}
	\end{minipage}
	\begin{minipage}{0.5\linewidth}
		\begin{stuki*}[5cm]{$\prodrule{U}{ \texttt{identifier} ~ \texttt{assign\_op} ~ E}$}
			\stm*[2]{
				\textit{E kódja} \\
				\texttt{mov [\textit{címke}], al}}
		\end{stuki*}
	\end{minipage}
	\caption{Értékadás kódgenerálási sémája}
\end{figure}

Elágazás kódgenerálási sémái. A kódgenerálási sémában szereplő címkék (pl. \textit{vége},
\textit{hamis}) helyett a séma minden felhasználásakor egyedi címkéket kell generálni (pl. \texttt{label0}, \texttt{label1}, \texttt{label2}, \dots)\footnote{Lásd a módosított UML-diagramot.}

\begin{figure}[h]
	\begin{minipage}{0.5\linewidth}
		\begin{stuki*}[5cm]{$\prodrule{U}{ \texttt{if} ~ E ~ \texttt{then} ~ P ~ \texttt{end}}$}
			\stm*[5]{
				\textit{E kódja} \\
				\texttt{cmp al, 1} \\
				\texttt{jne near \textit{vége}} \\
				\textit{P kódja} \\
				\texttt{\textit{vége}:}}
		\end{stuki*}
	\end{minipage}
	\begin{minipage}{0.5\linewidth}
		\begin{stuki*}[5cm]{$\prodrule{U}{ \texttt{if} ~ E ~ \texttt{then} ~ P_1 ~ \texttt{else} ~ P_2 ~ \texttt{end}}$}
			\stm*[7]{
				\textit{E kódja} \\
				\texttt{cmp al, 1} \\
				\texttt{jne near \textit{hamis}} \\
				\textit{$P_1$ kódja} \\
				\texttt{jmp \textit{vége}} \\
				\texttt{hamis:} \\
				\textit{$P_2$ kódja}
				\texttt{\textit{vége}:}}
		\end{stuki*}
	\end{minipage}
	\caption{Egy- és kétágú elágazás kódgenerálási sémája}
\end{figure}

\begin{figure}[h]
	\begin{minipage}{0.5\linewidth}
		\begin{stuki*}[5cm]{$\prodrule{U}{ \texttt{while} ~ E ~ \texttt{do} ~ P ~ \texttt{done}}$}
			\stm*[7]{
				\texttt{\textit{eleje}:} \\
				\textit{E kódja} \\
				\texttt{cmp al, 1} \\
				\texttt{jne near \textit{vége}} \\
				\textit{P kódja} \\
				\texttt{jmp \textit{eleje}} \\
				\texttt{\textit{vége}:}}
		\end{stuki*}
	\end{minipage}
	\begin{minipage}{0.5\linewidth}
		\begin{stuki*}[5cm]{$\prodrule{U}{ \texttt{do} ~ P ~ \texttt{while} ~ E}$}
			\stm*[5]{
				\texttt{\textit{eleje}:} \\
				\textit{P kódja} \\
				\textit{E kódja} \\
				\texttt{cmp al, 1} \\
				\texttt{je near \textit{eleje}}}
		\end{stuki*}
	\end{minipage}
	\caption{Elöl- és hátultesztelő ciklus kódgenerálási sémája}
\end{figure}

\pagebreak

Utolsó előtti lépésként a \textbf{szimbólumtáblát is rögzítenünk kell a kódban}. Ehhez végigiterálunk a tábla bejegyzésein és az \textbf{Assembly kód \texttt{.bss} szekciójába} beillesztjük. Sem a név, sem a méret miatt nem kell aggódnunk, ugyanis ezek az információk mind kinyerhetők a $label:\mathbb{S}$ (változónév) és $type:\text{Type}$ (méret) attribútumok segítségével.

\begin{figure}[h]
	\centering
	\begin{minipage}{0.5\linewidth}
		\begin{tabular}{|c|c|c|c|}
			\hline
			\textbf{Név} & \textbf{Fajta} & \textbf{Típus} & \textbf{Címke} \\
			\hline\hline
			\texttt{b} & változó & bool & \textit{lab0} \\
			\hline
			\texttt{i} & változó & int & \textit{lab1} \\
			\hline
		\end{tabular}
	\end{minipage}
	\begin{minipage}{0.3\linewidth}
		\begin{lstlisting}[style=asmstyle]
section .bss
lab0: resb 1
lab1: resd 1
		\end{lstlisting}
	\end{minipage}
\end{figure}

Végül, a teljes program sémája. Ebbe fognak beillesztődni az egyes kódsémák, kódrészletek, amint az elemző elérte a szintaxisfa gyökerét.

\begin{lstlisting}[style=asmstyle, caption={A teljes program sémája}]
global main
extern ; external labels

section .bss
; variables from symbol table

section .text
main:
; program instructions
ret
\end{lstlisting}