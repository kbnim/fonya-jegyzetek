\chapter{Lexikális elemzés}

\section{A tokenizáció}

Adott az alábbi karakterlánc:
\begin{center}
	\texttt{x = (x + 2) * 3;}
\end{center}
A feladat, hogy hogyan állapíthatjuk meg a benne lévő tokeneket?

Számunkra ránézésre nyilvánvaló, hogy a helyes tokenizáció a következő:
\begin{center}
	$\langle$ \texttt{x}, \texttt{=}, \texttt{(}, \texttt{x}, \texttt{+}, \texttt{2}, \texttt{)}, \texttt{*}, \texttt{3}, \texttt{;} $\rangle$.
\end{center}

Ezt kell valahogy a ``számítógép nyelvén'' kifejeznünk. Megállapítunk bizonyos tulajdonságokat, melyekkel kizárásos alapon kiválaszthatjuk a tokeneket.
\begin{itemize}
	\item \textbf{Aminek a belső szerkezete fontos, nem lehet token!} \\
	Például az értékadásnak van bal és jobb oldala, mindkettőnek megvannak a rá vonatkozó szabályai. 
	\begin{center}
		$\langle$ \texttt{x}, \texttt{=}, \texttt{(x + 2) * 3}, \texttt{;} $\rangle$ $\lightning$
	\end{center}
	
	Mivel az értékadás önmaga is egy utasítás, amire szintén vonatkoznak szabályok, emiatt az alábbi tokenizáció sem helyes.
	\begin{center}
		$\langle$ \texttt{x = (x + 2) * 3}, \texttt{;} $\rangle$ $\lightning$
	\end{center}
	
	\item \textbf{Aminek a formája nem írható le reguláris kifejezéssel, nem lehet token!} \\
	Tipikus példája ennek a helyes zárójelezések nyelve, ami környezetfüggetlen grammatikával írható le.
	\begin{center}
		$\langle$ \texttt{x}, \texttt{=}, \texttt{(x + 2)}, \texttt{*}, \texttt{3}, \texttt{;} $\rangle$ $\lightning$
	\end{center}
\end{itemize}

Következő probléma: a \textbf{fehérelválasztók} (\textit{whitespaces}). A legtöbb programozási nyelvnél a szóközök, tabulátorok és újsorok \textbf{nem alkotnak tokeneket}, csak más tokenek elválasztására valók. A lexikális elemzőnek fel kell ismernie ezeket, de nem kell továbbítania a szintaktikus elemző felé.

Ezalól kivételt képeznek a \textbf{behúzásra} (vagy indentációra) \textbf{érzékeny nyelvek}, mint a Python vagy a Haskell. Az elemzőnek a \textbf{sorok behúzását számon kell tartania}. Növekvő behúzás jelenti a blokknyitó tokent (C-ben \texttt{\{}), a csökkenő behúzás meg a blokkzáró tokent (C-ben \texttt{\}}).

Érdekességként megemlítjük a \href{https://www.wikiwand.com/en/Whitespace\_(programming\_language)}{\textit{Whitespace}} nyelvet, amiben kizárólag a fehérelválasztóknak van jelentése.

Ahogy korábban megállapítottuk, token csak az lehet, amit leírhatunk reguláris kifejezéssel. Bizonyos reguláris kifejezések elsőbbséget élveznek a többivel szemben -- nevezetesen azok, melyekkel kulcsszavakat írunk le. Tehát a konkrétabbak előrébb, az általánosabbak hátrébb kerülnek a felsorolásban.

\begin{figure}[h!]
	\centering
	\begin{tabular}{l|ll}
		%\hline
		\textbf{Reguláris kifejezés} & \textbf{Példák} & \textbf{Token típus} \\
		\hline
		\texttt{while} & \textit{while} & kulcsszó a While nyelvben \\
		%\hline
		\texttt{[a-zA-Z][a-zA-Z0-9\_]*} & \textit{x}, \textit{apple123}, \textit{list\_length} & azonosító \\
		%\hline
		\texttt{[+-]?[0-9]+} & \textit{0}, \textit{123}, \textit{-2}, \textit{+100} & egész számliterál \\
		%\hline
		\texttt{[ \textbackslash t\textbackslash n]+} & (fehérelválasztók nemüres sorozata) & -- \\
		%\hline
		\texttt{"//".*} & \textit{// Ez egy megjegyzés} & -- \\
		%\hline
	\end{tabular}
	\caption{Definíció reguláris kifejezéssekkel}
\end{figure}

\section{A lexikális elemzés elvei}

\begin{itemize}
	\item \textbf{Leghosszabb illeszkedés elve}\\
	A leghosszabban illeszkedő karaktersorozatból képzünk tokent.
	
	Pl. \texttt{w|hile}, \texttt{wh|ile}, \dots, \texttt{whil|e}$\to$ Hiába helyes azonosító szimbólum a \texttt{w}, \texttt{wh}, \dots, \texttt{whil}, mégis folytatni kell a keresést.
	
	\item \textbf{Prioritás elve} \\
	Ha a leghosszabban illeszkedő karaktersorozat több reguláris kifejezésre is illeszkedhet, a sorrendben korábban álló ``nyer''.
	
	Pl. \texttt{while|} $\to$ Lehet kulcsszó is, lehet azonosító is. Mivel kulcsszóként korábban definiáltuk, így ez élvez elsőbbséget.
\end{itemize}

\section{Implementációja}

A reguláris kifejezések átalakíthatók \textbf{véges determinisztikus automatá}vá.

\tikzset{
	->, % makes the edges directed
	%>=stealth’, % makes the arrow heads bold
	node distance=3cm, % specifies the minimum distance between two nodes. Change if necessary.
	%every state/.style={thick}, % sets the properties for each ’state’ node
	initial text=$ $ % sets the text that appears on the start arrow
}

\begin{minipage}{0.33\linewidth}
	\begin{center}
		\texttt{[a-z][a-z0-9]*}
	\end{center}
\end{minipage}
\begin{minipage}{0.66\linewidth}
	\begin{tikzpicture}
		%\node[state] (q1) {normal};
		%\node[state, initial, right of=q1] (q2) {start};
		%\node[state, accepting] at (1.5, 2) (q3) {accept};
		
		\node[state, initial] (q0) {$q_0$};
		\node[state, accepting, right of=q0] (q1) {$q_1$};
		\node[block, below of=q0, minimum width=0.7cm, minimum height=0.5cm] (error) {error};
		
		\draw (q0) edge[above] node{betű} (q1);
		\draw (q1) edge[loop right] node[align = left]{betű vagy \\ számjegy} (q1);
		\draw (q0) edge[right] node{nem betű} (error);
		\draw (q1) edge[right, bend left] node[align = left] {nem betű és \\ nem számjegy} (error);
	\end{tikzpicture}
\end{minipage}

A VDA implementációja történhet \textbf{elágazásokkal}, amelynek a struktogramja itt látható (a $\mathbb{K}$ jelöli a \texttt{char} adattípust).

\begin{stuki*}[14cm]{vda::process(next : $\mathbb{K})$}
	\begin{CASE}{2}{2}
		\WHEN{\stm{state = q_0}}
		\begin{IF}{1}{\stm{next = \texttt{'a'} \lor \dots}}
			\stm{state := q_1}
			\ELSE
			\stm{state := error}
		\end{IF}
		\WHEN{\stm{state = q_1}}
		\begin{IF}{1}{\stm{next = \texttt{'a'} \lor \dots \lor next = \texttt{'0'} \lor \dots}}
			\stm{state := q_1}
			\ELSE
			\stm{state := error}
		\end{IF}
	\end{CASE}
\end{stuki*}

\begin{lstlisting}[style=cppstyle, caption={VDA implementációja elágazásokkal}]
void vda::process(char next) 
{
	if (state == q0) {
		if (next == 'a' || ...) {
		    state = q1;
		} else {
		    state = error;
		}
	} else if (state == q1) {
		if (next == 'a' || ... || next == '0' || ...) {
		    state = q1;
		} else {
		    state = error;
		}
	}
}
\end{lstlisting}

Megoldható ugyanakkor \textbf{táblázattal} is.

\begin{figure}[h!]
	\centering
	\begin{tabular}{c||cc}
		%\hline
		& $q_0$ & $q_1$ \\
		\hline\hline
		\texttt{a} & $q_1$ & $q_1$ \\
		\hline
		\vdots & \multicolumn{2}{c}{\vdots} \\
		\hline
		\texttt{0} & error & $q_1$ \\
		\hline
		\vdots & \multicolumn{2}{c}{\vdots} \\
		\hline
		other & error & error \\
		%\hline
	\end{tabular}
	\caption{A VDA táblázata}
\end{figure}

\begin{stuki*}{vda::process$(next : \mathbb{K})$}
	\begin{IF}[80]{1}{\stm{state \neq error}}
		\stm{state := translations[state][next]}
		\ELSE
		\stm*{SKIP}
	\end{IF}
\end{stuki*}

\begin{lstlisting}[style=cppstyle, caption={VDA implementációja táblázattal}]
void vda::process(char next) 
{
	if (state != error) {
		state = transitions[state][next];
	}
}
\end{lstlisting}

%\textbf{\textit{Megjegyzés}}. A \texttt{vda::process} függvényben 

\newpage
\section{Tokenhez csatolt információk}

A felismert tokenekhez a lexikális elemző kiegészítő információkat csatol. Ezeket nevezzük \textbf{kitüntetett szintetizált attribútumok}nak. A jelentősségük a szemantikus elemzésnél fog megjelenni.

\begin{itemize}
	\item \textbf{Minden tokenhez}: a token pozícióját (első karakter sor- és oszlop-, utolsó karakter sor- és oszlopszáma)
	\item \textbf{Azonosítókhoz}: az azonosító szövegét (ez szükséges a szemantikus elemzéshez)
	\item \textbf{Literálokhoz}: a literál értékét (kódgeneráláshoz, kódoptimalizációhoz szükséges)
\end{itemize}

\section{Lexikális hibák}

Lexikális hiba esetén hibajelzést ad a fordító, és \textit{folytatja az elemzést}. A leggyakrabban előforduló hibák:
\begin{itemize}
	\item \underline{Illegális karakter}: A nyelv ábécéjébe nem tartozó karakter az inputszövegben. \\
	Az addig felépített token kiadja, ha volt illeszkedés. 
	Az illegális karaktert követő karakterrel folytatódik az elemzés.
	
	\item \underline{Lezáratlan sztring} \\
	A sor végén derül ki; az őt követő sorban folytatódik az elemzés.
	
	\item \underline{Lezáratlan többsoros megjegyzés} \\
	A fájl végén derül ki; nincs további elemzés.
\end{itemize}

%\section{Szemléltetés}

