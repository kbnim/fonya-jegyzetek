\chapter{Környezetfüggetlen (2-es típusú) nyelvtanok}

\section{A szóprobléma kérdése}

A \textit{formális nyelvek} témakörében az egyik központi kérdésünk, hogy adott $G$ grammatika és adott $u \in T^*$ szó estén teljesül-e, hogy $u \in L(G)$. Vagyis, hogy a $G$ nyelvtan által generált nyelvben benne van-e az $u$ szó. Szerencsére, a 2-es típusú nyelvtanok esetében vannak eszközeink ezen kérdésnek eldöntésére.

Ez az úgynevezett \textbf{szintaxisfa}, vagy \textbf{levezetési fa}. Az elnevezés több értelmet nyer, ha felidézzük a Backus--Naur-jelölést.

\begin{tcolorbox}
	\begin{definition}[Szintaxisfa]
		Legyen $G = (N,T,P,S) \in \mathcal{G}_2$ grammatika.
		A $t$ nemüres fát \textbf{$G$ feletti levezetési (szintaxis) fának}
		nevezzük, ha
		\begin{itemize}
			\item pontjai $T \cup N \cup \{\emptyword\}$ elemeivel vannak címkézve;
			\item belső pontjai $N$ elemeivel vannak címkézve;
			\item ha egy belső pont címkéje $A$, a közvetlen
			leszármazottjainak címkéi pedig balról jobbra olvasva
			$X_1, X_2, \dots, X_k$, akkor $A \longrightarrow X_1X_2\dots X_k \in P$.
			\item az $\emptyword$-nal címkézett pontoknak nincs testvére.
		\end{itemize}
	\end{definition}
\end{tcolorbox}

\begin{tcolorbox}
	\begin{theorem}
		Ha adott $G$ grammatika esetén $u \in L(G)$ akkor és csak
		akkor, ha $u$-hoz \textbf{megadható egy szintaxisfa}.
	\end{theorem}
\end{tcolorbox}

\textbf{\textit{Megjegyzés}}. Az $u$-hoz tartozó szintaxisfa gyökere $S$ és
a leveleit \textit{balról jobbra} összeolvasva az $u$ szót kapjuk.

\begin{tcolorbox}
	\begin{proposition}
		Minden szintaxisfához megadható egy levezetés és fordítva.
	\end{proposition}
\end{tcolorbox}

\begin{tcolorbox}
	\begin{definition}[Egyértelmű nyelvtan]
		Egy $G \in \mathcal{G}_2$ nyelvtan \textbf{egyértelmű}, ha minden $u \in L(G)$ szóhoz egyetlen szintaxisfa tartozik.
	\end{definition}
\end{tcolorbox}

%Megelőlegezzük, hogy \textbf{a szóprobléma eldönthető az összes környezetfüggetlen nyelvtan által generált nyelvre}. Ennek eldöntésére szintén egy új automatát vezetünk be, az ún. \textbf{veremautomatá}kat. Róluk pár alfejezettek később lesz bővebben szó.

\section{2-es típusú nyelvtanok normálformája}

A Chomsky-hierarchia bevezetésénél kimondtuk az \textbf{$\emptyword$-mentesítés} tételét, amit 2-es és 3-as típusú nyelvtanokon elvégezhető transzformáció. 

\begin{tcolorbox}
	\begin{definition}[Chomsky-normálforma]
		Egy $G=(N,T,P,S) \in \mathcal{G}_2$ %környezetfüggetlen
		nyelvtant \textbf{Chomsky-normálformájú}nak
		mondunk, ha szabályai
		\begin{itemize}
			\item \framebox{$\prodrule{A}{a}$} , ahol $A \in N$ és $a \in T$ vagy
			\item \framebox{$\prodrule{A}{BC}$} alakúak, ahol $A, B, C \in N$.
			\item \framebox{$\prodrule{S}{\emptyword}$} , de ekkor $S$ nem fordul elő egyetlen
			szabály jobboldalán sem.
		\end{itemize}
	\end{definition}
\end{tcolorbox}

\begin{tcolorbox}
	\begin{theorem}	
		Minden \textbf{környezetfüggetlen} grammatikához
		megkonstruálható egy vele ekvivalens \textbf{Chomsky
		normálformájú} grammatika.
	\end{theorem}
\end{tcolorbox}

\textbf{\textit{Megjegyzések}}.
\begin{itemize}
	\item A 2-es típusú grammatikák Chomsky normálformára hozásának algoritmusa nem a
	tananyag része.\footnote{A jegyzet írása idején így szólt a tanterv (2023/2024/2. félév).}
	\item Chomsky normálformájú grammatikákhoz megadható olyan elemző program,
	amely $O(n^3)$ időben eldönti a szóproblémát (\textit{Cocke--Younger--Kasami-algoritmus}).
	\item Bizonyos állítások bizonyítását elég elvégezni a normálformájú grammatikákra.
\end{itemize}

A korábban 3-as típusú nyelvtanokra megfogalmazott \textit{kis Bar-Hillel lemmá}t megfogalmazzuk környezetfüggetlen nyelvekre is.

\begin{tcolorbox}
	\begin{lemma}[Nagy Bar-Hillel lemma]
		$\forall L \in \mathcal{L}_2$ nyelvhez $\exists p, q \in \mathbb{N}$ \textbf{nyelvfüggő
		konstans}, hogy $\forall u \in L$, ahol $\ell(u) \geq p$ szó esetén van
		$u$-nak olyan $u = vxwyz$ felbontása $(v, x, w, y, z \in T^*)$, amelyre
		\begin{itemize}
			\item $\ell(xwy) \leq q$,
			\item $xy \neq \emptyword$,
			\item $\forall i \in \mathbb{N} : vx^iwy^iz \in L$.
		\end{itemize}
	\end{lemma}
\end{tcolorbox}

\textbf{\textit{Megjegyzések}}.
\begin{itemize}
	\item A lemma \textbf{szükséges feltétel} környezetfüggetlen nyelvekre.
	\item A lemmát nem bizonyítjuk, de a bizonyításhoz szükséges, hogy a
	2-es típusú nyelvekhez létezik Chomsky-normálformájú grammatika.
\end{itemize}

\textbf{\textit{Következmény}}. Van olyan nyelv, amely nem környezetfüggetlen. Például
\[ L := \{ \texttt{a}^n\texttt{b}^n\texttt{c}^n \mid n > 0 \} \notin \mathcal{L}_2. \]
Tegyük fel indirekt, hogy $\exists p, q$ a Bar-Hillel lemmának megfelelő konstansok. Legyen $k > p$ és $k > q$ is. Ekkor $u = \texttt{a}^n\texttt{b}^n\texttt{c}^n$ és $\ell(u) > p$.

A lemma szerint az $u$ szó $vxwyz$ alakban felbontható kell legyen, úgy, hogy $\ell(xwy) \leq q <k$ és $x$ és $y$ párhuzamosan beiterálható. De ekkor $xy$-ban nem lehet mindhárom betűből, így $vwz$ nem
lehet eleme $L$-nek, ami ellentmondás.

\section{Nyelvtan redukálása}

A grammatikák transzformálása közben keletkezhetnek olyan szabályok, amelyek egyetlen szó levezetésében sem használhatóak.

A grammatikában lehetnek olyan nemterminálisok, amelyekből

\begin{enumerate}
	\item nem lehet csupa nem terminálisból álló sorozatot
	előállítani (\textit{zsákutcák});
	\item nem érhetők el a kezdőszimbólumból.
\end{enumerate}

\begin{tcolorbox}
	\begin{definition}[Aktív nemterminálisok]	
		\textbf{Aktív} nemterminálisok halmaza egy adott $G=(N,T,P,S)\in \mathcal{G}_2$ grammatika esetén:
		\[ A := \left\{ X \in N \setdivbar X \genword{G}{*} u \land u \in T^* \right\}. \] 
	\end{definition}
\end{tcolorbox}

Inaktív (\textbf{zsákutca}) nemterminálisok: $\boxed{N \setminus A}$ .

\begin{tcolorbox}
	\begin{definition}[Elérhető nemterminálisok]	
		\textbf{Elérhető} nemterminálisok halmaza egy adott $G=(N,T,P,S)\in \mathcal{G}_2$ grammatika esetén:
		\[ R := \left\{ X \in N \setdivbar S \genword{G}{*} uXw \land u, w \in (T \cup N)^* \right\}. \] 
	\end{definition}
\end{tcolorbox}

Nem elérhető nemterminálisok: $\boxed{N \setminus R}$ .

\begin{tcolorbox}
	\begin{definition}[Hasznos nemterminálisok]	
		Egy nemterminálist \textbf{hasznos}nak mondunk, ha \textbf{aktív és elérhető}:
		\[ X \in (A \cap R) \subseteq N. \]
	\end{definition}
\end{tcolorbox}

\begin{tcolorbox}
	\begin{definition}[Redukált nyelvtan]	
		Egy környezetfüggetlen grammatika
		\textbf{redukált}, ha minden nemterminálisa \textbf{hasznos}, azaz
		a grammatika \textbf{zsákutcamentes és összefüggő}.
	\end{definition}
\end{tcolorbox}

\begin{tcolorbox}
	\begin{theorem}
		Minden $G \in \mathcal{G}_2$ nyelvtanhoz
		megkonstruálható egy vele ekvivalens \textbf{redukált grammatika}.
	\end{theorem}
\end{tcolorbox}

\textbf{\textit{Bizonyítás}}. // Kidolgozni.

\begin{enumerate}
	\item Zsákutcák meghatározása és minden olyan szabály
	elhagyása, amiben inaktív nemterminálisok szerepelnek.
	\item Az $S$-ből nem elérhető nemterminálisokhoz tartozó
	szabályok elhagyása, azaz a grammatika összefüggővé
	tétele.
\end{enumerate}

\section{Veremautomaták}

Elöljáróban elárultuk, hogy a szóprobléma eldönthető az összes környezetfüggetlen nyelvtan által generált nyelvre. Íme az erre vonatkozó tétel és bizonyítása.

\begin{tcolorbox}
	\begin{theorem}[Szóprobléma eldöntése]
		Minden $G=(N,T,P,S)\in \mathcal{G}_2$ grammatika esetében eldönthető, hogy
		egy tetszőleges $u \in T^*$ szó benne van-e a $G$
		grammatika által generált nyelvben vagy sem.
	\end{theorem}
\end{tcolorbox}

\textbf{\textit{Bizonyítás}}. // Kidolgozni.

A szóprobléma eldönthető a környezetfüggetlen grammatikákhoz párosuló \textbf{veremautomatá}kkal, melyet az alábbi módon definiálunk.

\begin{tcolorbox}
	\begin{definition}[Veremautomata] \textbf{Veremautomatá}nak nevezzük az
		\[ A = (Z, Q, T, \delta, z_0, q_0, F) \]
		rendezett hetest, ahol
		\begin{itemize}
			\item $Z$ a verem szimbólumainak ábécéje,
			\item $Q$ az állapotok halmaza $(0 < |Q| < \infty)$,
			\item $T$ a bemeneti szimbólumok ábécéje,
			\item $\delta : Z \times Q \times (T \cup \{\emptyword\}) \to \mathcal{P}(Z^* \times Q)$ leképezés az állapotátmeneti függvény, ahol $\delta$ véges részhalmazokba képez,
			\item $z_0 \in Z$ a kezdő veremszimbólum,
			\item $q_0 \in Q$ a kezdőállapot
			\item $F \subseteq Q$ az elfogadóállapotok halmaza.
		\end{itemize}
	\end{definition}
\end{tcolorbox}

Egy lépésben mindig kell egy jelet olvasni a verem tetejéről és csak egy jelet lehet
elérni. Az input szalagról is egy jelet lehet olvasni, de nem kötelező.

Megváltoztatható az automata aktuális állapota, illetve a verem teteje. Egy lépésben
egy egész sorozatot is beírhatunk a verembe.

\textbf{\textit{Példák}}.
\begin{itemize}
	\item \framebox{$\delta(\#,q,a) = \{(\#a,q)\}$} \\ \textit{Jelentése}: Ha \# van a verem tetején és $a$ betű jön az inputon, akkor \textbf{tegyük be} $a$-t a verembe. Ne változtassunk az állapoton.
	
	\item \framebox{$\delta(\#,q,a) = \{(\emptyword,q)\}$} \\ \textit{Jelentése}: Ha \# van a verem tetején és $a$ betű jön az inputon, akkor \textbf{töröljük} \#-t a veremből. Ne változtassunk az állapoton.
	
	\item \framebox{$\delta(\#,q,a) = \{(\#,r)\}$} \\ \textit{Jelentése}: Ha \# van a verem tetején és $a$ betű jön az inputon, akkor \textbf{ne változtassuk a verem tartalmát}. Viszont váltsunk állapotot. 
	
	\item \framebox{$\delta(\#,q,\emptyword) = \{(\#bb,q)\}$} \\ \textit{Jelentése}: Ha \# van a verem tetején és nem olvasunk az inputról, akkor \textbf{tegyünk a verembe} két $b$ betűt és váltsunk állapotot is.
\end{itemize}

Ahogyan azt a véges automatáknál is tapasztalhattuk, létezik az állapotátmeneteknek egy alternatív jelölése, mely követi azt a konvenciót, ahogyan a nyelvtani szabályokat írjuk fel.

\begin{itemize}
	\item Ha $\delta(z,q,a) = \{(w_1,r_1),\dots,(w_k,r_k)\}$ $(k \in \mathbb{N}^+)$ , akkor ezt a leképezést a következő szabályhalmazzal is jelölhetjük:
	\[ \prodrule{zqa}{w_i r_i} ~~~~ i \in [1 \dotdot k]. \]
	\item Ha $\delta(z,q,\emptyword) = \{(w_1,r_1),\dots, (w_k,r_k)\}$ $(k \in \mathbb{N}^+)$, akkor ezt a leképezést a következő szabályhalmazzal is jelölhetjük:
	\[ \prodrule{zq}{w_i r_i} ~~~~ i \in [1 \dotdot k]. \]
\end{itemize}

Tehát a szabályok bal oldala $ZQT$ vagy $ZQ$ alakú és a jobboldala $Z^*Q$ alakú.

Ahogyan a véges automatáknál, itt is értelmezzük a \textbf{konfiguráció} fogalmát -- természetesen a megfelelő módosításokkal.

\begin{tcolorbox}
	\begin{definition}[Konfiguráció]
		Legyen $A = (Z, Q, T, \delta, z_0, q_0, F)$ egy veremautomata és
		legyen $\alpha \in Z^*QT^*$.
		Azt mondjuk $\alpha$ az $A$ veremautomata egy \textbf{konfigurációja}.
	\end{definition}
\end{tcolorbox}

Szavakban: a konfiguráció a veremautomata egy pillanatnyi állapotát írja le.

Hasonlóan, a redukciót fogalmát is értelmezzük. Ahogy azt a véges automatáknál is tapasztalhattuk, a ``sima'' redukció magába foglalja a közvetett redukciót, így a kettőt nem szoktuk megkülönböztetni.

\begin{tcolorbox}
	\begin{definition}[Közvetlen redukció]
		Legyen $A = (Z, Q, T, \delta, z_0, q_0, F)$ egy veremautomata és
		legyen $\alpha, \beta \in Z^*QT^*$ konfigurációk.
		
		Azt mondjuk, hogy az A veremautomata az $\alpha$ konfigurációt
		a $\beta$ konfigurációra \textbf{redukálja közvetlenül}, ha
		\[ \exists z \in Z, p, q \in Q, a \in T\cup\{\emptyword\}, r,u \in Z^*, w \in T^*, \text{ hogy } \]
		\begin{itemize}
			\item \framebox{$\prodrule{zqa}{up}$} egy szabály $\delta$-ban,
			\item $\alpha = r\textbf{zqa}w$ és $\beta = r\textbf{up}w$.
		\end{itemize}
		~
		
		Jele: \framebox{$\alpha \genword{A}{} \beta$} .
	\end{definition}
\end{tcolorbox}

A vastag betűk továbbra is arra szolgálnak, hogy szemléletesebbé váljon a helyettesítés és nem vektorokat jelentenek.

\begin{tcolorbox}
	\begin{definition}[Redukció]
		Legyen $A = (Z, Q, T, \delta, z_0, q_0, F)$ egy veremautomata és
		legyenek $\alpha, \beta \in Z^*QT^*$ konfigurációk.
		
		Azt mondjuk, hogy az A veremautomata az $\alpha$ konfigurációt
		a $\beta$ konfigurációra \textbf{redukálja közvetetten}, 
		\begin{itemize}
			\item ha vagy $\alpha = \beta$,
			\item vagy ha $\alpha \neq \beta$, akkor $\exists \alpha_1 \dots \alpha_k$ $(k \in \mathbb{N}^+)$ konfiguráció sorozat, hogy $\alpha_1 = \alpha$ és $\alpha_k = \beta$ és $\forall i \in [1 \dotdot k-1] : \alpha_i \genword{A}{} \alpha_{i+1}$.
		\end{itemize}
		
		Jele: \framebox{$\alpha \genword{A}{*} \beta$} .
	\end{definition}
\end{tcolorbox}

A véges automatákkal szemben itt kétféle ``nyelvet'' értelmezünk attól függően, hogy megen-gedjük-e, hogy a verem lehet-e üres.

\begin{tcolorbox}
	\begin{definition}[Elfogadó állapottal felismerhető nyelv]
		\[ L(A) := \left\{  u \in T^* \setdivbar \exists z_0 q_0 u \genword{A}{*} wr \land r \in F \land w \in Z^* \right\} \]
	\end{definition}
\end{tcolorbox}

Ez azt jelenti, hogy van olyan működése a veremautomatának, hogy
kezdő konfigurációból indulva végig olvasva az inputot elfogadóállapotba jut.

\begin{tcolorbox}
	\begin{definition}[Üres veremmel felismerhető nyelv]
		\[ N(A) := \left\{  u \in T^* \setdivbar \exists z_0 q_0 u \genword{A}{*} r \land r \in F \right\} \]
	\end{definition}
\end{tcolorbox}

Ez azt jelenti, hogy van olyan működése a veremautomatának, hogy
kezdő konfigurációból indulva végig olvasva az inputot teljesen kiüríti a vermet.

A veremautomaták \textbf{determinisztikusságát} is vizsgálhatjuk.

\begin{tcolorbox}
	\begin{definition}[Determinisztikus veremautomata]
		Egy veremautomatát \textbf{determinisztikus}nak mondunk, ha
		$\forall \alpha \in Z^+QT^*$ konfiguráció esetén egyetlen
		konfiguráció vezethető le közvetlenül $\alpha$-ból.
	\end{definition}
\end{tcolorbox}

Ez azt jelenti, hogy nincs két olyan szabály, amelynek azonos a
bal oldala, valamint, ha $zq$ egy bal oldal, akkor nincs $zqa$ bal oldal egyetlen
terminálisra sem.

A determinisztikus veremautomatával felismerhető nyelvek családja szűkebb, mint a nemdeter-
minisztikussal felismerhető nyelvek családja. Például a szimmetrikus szavak nem ismerhetők fel
determinisztikus veremautomatával.

\begin{tcolorbox}
	\begin{lemma}
		Bármely $A$ veremautomatához megadható $A'$
		veremautomata úgy, hogy $N(A')=L(A)$.
	\end{lemma}
\end{tcolorbox}

\begin{tcolorbox}
	\begin{lemma}
		Bármely $A$ veremautomatához megadható $A'$
		veremautomata úgy, hogy $L(A')=N(A)$.
	\end{lemma}
\end{tcolorbox}

\textbf{\textit{Megjegyzés}}. Ez azt jelenti, hogy ha egy nyelvhez építhető elfogadó
állapottal felismerő veremautomata, akkor építhető üres veremmel felismerhető
veremautomata és fordítva.

\begin{tcolorbox}
	\begin{theorem}
		Minden $L \in \mathcal{L}_2$ nyelvhez megadható egy $A$ veremautomata
		úgy, hogy
		\[ L = N(A), \text{ ~ azaz ~ } \mathcal{L}_2 \subseteq \mathcal{L}_{1V}. \]
	\end{theorem}
\end{tcolorbox}

\textbf{\textit{Bizonyítás}}. // Kidolgozni.

\begin{tcolorbox}
	\begin{theorem}
		Minden $A$ veremautomatához megadható egy
		$G \in \mathcal{G}_2$ nyelvtan úgy, hogy
		\[ L(G) = N(A), \text{ ~ azaz ~ } \mathcal{L}_{1V} \subseteq \mathcal{L}_{2}. \]
	\end{theorem}
\end{tcolorbox}

A fordított tételt nem bizonyítjuk.

\iffalse

\section{Feladatok}

\subsection{Környezetfüggetlen nyelvek normálformára hozása}

\begin{enumerate}
	\item tartalom...
\end{enumerate}

\subsection{Veremautomaták}

\begin{enumerate}
	\item tartalom...
\end{enumerate}

\fi