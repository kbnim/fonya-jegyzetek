\chapter{Szemantikus elemzés}

A szintaktikus elemzés létrehozza a szintaxisfát. A szemantikus ellenőrzés azt állapítja meg róla, hogy ``van-e értelme'' annak, amit kifejez -- mindezt fordítási időben.

Nyelvtől függően jelentősen eltérhetnek a specifikus feladatai, így csak általánosságokban fogunk róluk értekezni. A szemantikus elemzés két eszközt használ: ezek a \textbf{szimbólumtábla} és az \textbf{attribútumnyelvtan}.

\section{Szimbólumtábla}

A szimbólumtábla a deklarációkat tárolja. A fordítóban gyakran globális változó. Segítségével a szemantikus elemzés
\begin{itemize}
	\item feldolgozza a deklarációkat,
	\item az azonosítószimbólumokat deklarációhoz köti,
	\item ellenőrzi a hatókörrel és láthatósággal kapcsolatos szabályokat.
\end{itemize} 

A következő, \textbf{tipikus hibák}at képes kiszűrni:
\begin{itemize}
	\item deklarálatlan változókat,
	\item újradeklarált változókat,
	\item változó hatókörön kívüli használatát,
	\item privát adattagok elérését (pontosabban azoknak a korlátozását),
	\item elfedésből adódó típushibákat.
\end{itemize}

Ahogy azt megtárgyaltuk, a lexikális elemző a forráskód karaktersorozatából tokenek sorozatát állítja elő.
\begin{center}
	\texttt{x = (x + 2) * 3;}
	
	$\downarrow$
	
	$\langle$ \texttt{x}, \texttt{=}, \texttt{(}, \texttt{x}, \texttt{+}, \texttt{2}, \texttt{)}, \texttt{*}, \texttt{3}, \texttt{;} $\rangle$.
\end{center}

Arról is beszéltünk, hogy egyes tokenek rendelkezhetnek kiegészítő infromációkkal (az azonosító a szövegüket, a literálok az értéküket, stb.). Megelőlegezzük, hogy {kitüntetett szintetizált attribútumok}nak hívjuk őket.

\begin{figure}
	\centering
	%\typedef{$\mathbb{N} \times \mathbb{N}$}{Coordinate}
	
	\enum{Kind}{{function}, {parameter}, {local variable}, stb.}
	
	\enum{Type}{int, int$\to$void, stb.}
	
	\struct{SymbolTable}
	+ $name : \mathbb{S}$ \\
	+ $kind : Kind$ \\
	+ $type : Type$ \\
	+ $declaration : \mathbb{N} \times \mathbb{N}$ ~ \texttt{/* (row, column) */} \\
	+ $used\_here : \mathbb{N} \times \mathbb{N}\langle\rangle$ \texttt{/* list of coordinates */}\\
	\texttt{/* additional fields may come here */} \\
	\hline
	+ SymbolTable(\dots) \\
	\texttt{/* additional methods may come here */} \\
	\eoStruct
	\caption{A szimbólumtábla egy lehetséges megvalósítása}
	\label{symboltabletypedef}
\end{figure}

Egy egyszerűbb nyelv esetében a szimbólumtáblát implementálhatjuk egy \framebox{\textbf{hasító táblá}val}. Ez olyan összetett típusú objektumokat tartalmaz, amelyek az adott deklarált ``egységről'' (legyen az változó, függvény, stb.) információkat tárolnak, mint például

\begin{itemize}
	\item a nevét,
	\item fajtáját (függvény, függvényparaméter, lokális vagy globális változó, ciklus, elágazás, névtelen blokk, stb.),
	\item típusát (egész szám, int$\to$void típusú függvény, stb.)
	\item a deklarációja helyét (a nevének első karaktere az eredeti szövegben melyik sor melyik oszlopában történik),
	\item mikor használtuk a program során.
\end{itemize}

\textbf{Két művelet}tel is rendelkezik.

\begin{enumerate}
	\item \underline{Beszúrás}
	
	\begin{itemize}
		\item deklaráció esetén
		\item az új szimbólum és adatai bekerülnek a szimbólumtáblába
		\item a beszúrás \textit{mindig egy kereséssel kezdődik}, hogy kiderüljön az újradeklarálás
	\end{itemize}
	
	\item \underline{Keresés}
	
	\begin{itemize}
		\item szimbólum használatakor
		\item a szimbólum \textit{neve a kulcs} a kereséshez
		\item a szimbólum használatát érdemes feljegyezni (pl. refaktoráláshoz)
	\end{itemize}
\end{enumerate}

\begin{figure}[h!]
	\begin{minipage}{0.275\linewidth}
		\begin{lstlisting}[style=cppstyle]
void f(int p) {
	int x;
	cin >> x;
	cout << x+p+y;
}
		\end{lstlisting}
	\end{minipage}
\begin{minipage}{0.01\linewidth}
	~
\end{minipage}
	\begin{minipage}{0.5\linewidth}
			\begin{tabular}{|c|c|c|c|c|}
				\hline
				\textbf{Név} & \textbf{Fajta} & \textbf{Típus} & \textbf{Deklaráció} & \textbf{Használat} \\
				\hline\hline
				\texttt{f} & függvény & int$\to$void & $(1,6)$ & $\langle\rangle$ \\
				\hline
				\texttt{p} & paraméter & int & $(1,12)$ & $\langle(4,13)\rangle$ \\
				\hline
				\texttt{x} & lokális változó & int & $(2,7)$ & $\langle(3,10), (4,11)\rangle$ \\
				\hline
			\end{tabular}
	\end{minipage}
	\caption{A kódrészlet és a hozzá tartozó szimbólumtábla}
\end{figure}

\textbf{\textit{Megjegyzés.}} Ha a korábbi kódrészletben a függvény végére beillesztenénk a(z)
\[\texttt{int x;}\] sort, akkor a változót nem tudná beszúrni a táblába, hiszen szerepel már egy ilyen nevű és típusú változó. Ilyenkor a compiler újradeklarálás hibájával fog visszajelezni. 

Jobban járunk, ha hasító tábla helyett \framebox{\textbf{verem adatszerkezet}et} használunk a szimbólumtáblához. Ez sokkal jobb megoldás, ugyanis neki köszönhetően képesek vagyunk kezelni a \textbf{blokkszerkezetek}et, a változók \textbf{hatókör}ét, \textbf{láthatóság}át, valamint az \textbf{elfedés}t is!

A beszúrás lecserélődik egy klasszikus $push()$ műveletre.
Keresékor fentről lefelé keresünk a veremben, és az \textit{első találat}nál megállunk.

Továbbá felveszünk egy ún. \textbf{blokk-index vektor}t, ami a nevével ellentétben szintén egy \textbf{verem}. Ennek az a feladata, hogy \textbf{amikor új blokk kezdődik, a blokk-index vektorban megjelöljük a szimbólumtábla-verem tetejét} -- magyarán egy olyan pointert push-olunk bele, ami az adott ``blokkelemre'' mutat a szimbólumtáblában\footnote{A 9.3. ábrában a \& jel jelöli az \texttt{f} nevű függvényhez tartozó szimbólumobjektum memóriacímét.}. Függvényhívás esetén megfeleltethető a függvény aktivációs rekordjának. \textbf{Minden blokk megkezdésekor új mutató kerül a vektor tetejére}. A blokk végén eltávolítjuk a szimbólumokat a blokk-index vektor legfelső jelöléséig (néhány $pop()$). Végül a jelölést is eltávolítjuk a blokk-index vektorból.

\begin{figure}[h!]
		\begin{lstlisting}[style=cppstyle]
int x = 1;
void f(int p) { // <- semantic analysis in progress
	cout << x;
	int x = 2;
	cout << x+p;
}
int p = x;
		\end{lstlisting}

	\begin{minipage}{0.6\linewidth}
		\centering
		\begin{tabular}{|c|c|c|c|}
			\hline
			\vdots & \vdots & \vdots & \vdots \\
			\hline
			\texttt{p} & paraméter & int & $(2,12)$ \\
			\hline
			\texttt{f} & függvény & int$\to$void & $(2,6)$ \\
			\hline
			\texttt{x} & globális változó & int & $(1,5)$  \\
			\hline\hline
			\textbf{Név} & \textbf{Fajta} & \textbf{Típus} & \textbf{Deklaráció} \\
			\hline
		\end{tabular}
	\end{minipage}
	\begin{minipage}{0.4\linewidth}
		\centering
		\begin{tabular}{|c|}
			\hline
			 \\
			\hline
			\\
			\hline
			\\
			\hline
			\&(\texttt{f}) \\
			\hline\hline
			\textbf{Blokk-index vektor} \\
			\hline
		\end{tabular}
	\end{minipage}
	\label{symboltablestack}
	\caption{Verem szerkezetű szimbólumtábla}
\end{figure}

Deklaráció feldolgozásakor ellenőrizni kell, hogy nem újradeklarált változóról van-e szó. Csak ez után szabad beszúrni a szimbólumot és adatait a táblázatba. A verem szerkezetű szimbólumtáblában \textbf{csak a blokk-index vektor legfelső bejegyzése által mutatott rekord fölött \underline{keresünk}}, azaz az aktuális blokk szimbólumai között. Ha a blokk-index vektor üres, akkor az egész szimbólumtáblában keresünk. Ha nincs hiba, a szimbólum beszúrható a táblába.

\section{Attribútumnyelvtan}

A típusrendszerek a programhibák felderítésének legfontosabb eszközei. Ezek jelölik ki, milyen műveletek végezhetők az adatokkal. Vannak a jól ismert alaptípusaink (bool, int, char, stb.), de a nyelv támogathat összetett típusokat is (tömb, rekord, mutató, referencia, osztály, interfész, unió, algebrai típusok, stb.). Ugyanakkor léteznek típus nélküli nyelvek is -- ilyen a legtöbb Assembly nyelv.

%Ha a nyelvünk típusokkal rendelkezik, akkor attribútumnyelvtanra lesz szükségünk. Ennek segítségével végezhetünk típusellenőrzést, típuslevezetést, valamint típuskonverziókat.

Az \textbf{ellenőrzés két fázis}ban történik.

\begin{enumerate}
	\item \underline{Statikus típusozás} -- fordítási időben
	\begin{itemize}
		\item Futás közben már \textbf{csak az értékeket kell tárolni}, típusinformációt nem
		\item Ha a program lefordul, típusokkal kapcsolatos hiba nem történhet futás közben: \textbf{biztonságosabb} megoldás
		\item Ada, C++, Haskell, stb.
	\end{itemize}
	\item \underline{Dinamikus típusozás} -- futási időben
	\begin{itemize}
		\item Futási időben az értéket mellett \textbf{típusokat is kell tárolni}
		\item Az utasítások \textbf{végrehajtása előtt kell ellenőrizni} a típusokat
		\item Futás közben derülnek ki a típushibák, cserébe \textbf{hajlékonyabbak} az ilyen nyelvek
		\item Lisp, Erlang, stb.
	\end{itemize}
\end{enumerate}

A statikusan típusos nyelvek is használnak dinamikus technikákat: dinamikus kötés, Java \texttt{instanceof} operátora.

A típusokat is kétféleképpen adhatjuk meg.
\begin{enumerate}
	\item \underline{Programozó adja meg} $\to$ típusellenőrzés
	\begin{itemize}
		\item A deklarációk típusozottak
		\item A kifejezések egyszerű szabályok alapján típusozhatók
		\item Egyszerűbb fordítóprogram, gyorsabb fordítás
	\end{itemize}
	\item \underline{Fordítóprogram találja ki} $\to$ típuslevezetés, típuskikövetkeztetés
	\begin{itemize}
		\item A deklarációkhoz (általában) nem kell típust megadni
		\item A kifejezések típusát a fordítóprogram ``találja ki'' a műveletek alapján
		\item Kényelmesebb a programozónak -- azonban ajánlott típusozni a deklarációkat, hogy olvashatóbb legyen a
		kód
	\end{itemize}
\end{enumerate}

A \textbf{típuskonverzió} a kifejezés típusának megváltoztatását jelenti. Ez történhet automatikusan vagy explicit konverzióval is (C/C++-ban ez a kasztolás). A fordítóprogramnak ügyelnie kell az osztályhierarchiához kapcsolódó típuskonverziókra -- azaz a \textit{Liskov-féle helyettesítési elv}re. A típuskonverziókkal a kódgenerátornak is törődnie kell: adatkonverzióra is szükség lehet.

A szintaxist leíró nyelvtan szimbólumaihoz \textbf{attribútumok}at rendelünk, melyek a szemantikus elemzés vagy a kódgenerálás, kódoptimalizálás számára fontos, kiegészítő információk. Az ilyen nyelvtant nevezzük \textbf{attribútumnyelvtan}nak. A szabályokhoz \textbf{akciók}at (programkód részleteket) rendelünk. Ezek a meglévő attribútumértékekből újabb attribútumok értékeit számolják ki, valamint ellenőrzéseket végeznek, szemantikus hibákat jeleznek.

Legyen az elemzendő szövegünk a(z) \framebox{\texttt{10 < x}} az alábbi szintaxisfával \\ (a nyelvtan: $\prodrule{E}{\texttt{int\_literal} \mid \texttt{identifier} \mid E ~ \texttt{less\_than} ~ E}$).

\begin{figure}[h]
	\centering
	\begin{forest}
		for tree={edge={-}, l sep+=20pt, s sep+=10pt}
		[$E$
		[$E$
		[\texttt{int\_literal}, tier=terminal]
		]
		[\texttt{less\_than}, tier=terminal]
		[$E$
		[\texttt{identifier}, tier=terminal]
		]
		]
	\end{forest}
\end{figure}

A következő attribútumokra lesz szükségünk.
\begin{itemize}
	\item Az azonosítóknak a nevére, amit $\texttt{identifier}.name$ formában érhetünk el (lásd a \ref{symboltabletypedef}. ábrát). Ezek a \textbf{kitüntetett szintetizált attribútum}ok.
	\item A kifejezésekhez is hozzárendelünk (egész pontosan a \textit{szabály bal oldalához}) egy $type:Type$ attribútumot, amit $E.type$ formában érhetünk el. Az ilyen attribútumokat \textbf{szintetizált attribútum}oknak nevezzük. Az $LR$ elemzőkhöz nagyon jól illeszkednek.
\end{itemize}

Az elemző \textbf{alulról felfele} haladva meghatározza először a literálok típusát. A felsőbb szintekhez érve a korábbiak alapján meghatározza az egész kifejezés típusát is az \textbf{akciók} segítségével. 

%A kifejezésekhez most meghatározzuk az \textbf{akciókat}, azaz a kódrészleteket, amik ellenőrzéseket hajtanak végre, amik siker esetén hozzárendelik a típust.

%Meghatározzuk, hogy az $E$ kifejezés rendelkezik egy $type : Type$ adattaggal, valamint az \texttt{identifier} nemterminális tokenből elérhetjük az azonosító nevét: $\texttt{identifier}.name$.

\begin{stuki*}[14cm]{$\prodrule{E}{\texttt{int\_literal}}$}
	\stm[3]{E.type := \text{int} \\ 
	\text{/* Akció, amely az egyetlen
		egészszám-literálból} \\ \text{álló
		kifejezés típusát
		számolja ki. */}}
\end{stuki*}

\begin{stuki*}[14cm]{$\prodrule{E}{\texttt{identifier}}$}
	\stm{\text{/* Ellenőrzés: Deklarálva volt-e az azonosító? */}}
	\begin{IF}[20]{2}{\stm{\texttt{identifier}.name \notin symbol\_table}}
		\stm{error(\dots)}
		\ELSE
		\stm[2]{E.type := symbol\_table.get\_type(\texttt{identifier}.name) \\
		\text{/* Azonosító kifejezés típusának beállítása */}}
	\end{IF}
	%\stm{E.type := \text{int}}
\end{stuki*}

\begin{stuki*}[14cm]{$\prodrule{E_1}{E_2 ~ \texttt{less\_than} ~ E_3}$}
	\stm{\text{/* Ellenőrzés: Megfelelő típusúak-e a ``\texttt{<}'' operátor paraméterei? */}}
	\begin{IF}[20]{2}{\stm{E_2.type \neq \text{int} \lor E_3.type \neq \text{int}}}
		\stm{error(\dots)}
		\ELSE
		\stm[2]{E_1.type := \text{bool} \\ \text{/* Az összes kifejezés típusának kiszámítása */}}
	\end{IF}
	%\stm{E.type := \text{int}}
\end{stuki*}

Az újonnan megállapított szintetikus attribútumok így bekerülnek a szintaxisfába.

\begin{figure}[h]
	\centering
	\begin{forest}
		for tree={edge={-}, l sep-=10pt, s sep+=10pt}
		[$E$\textsuperscript{$type : \text{bool}$}
		[$E$\textsuperscript{$type : \text{int}$}
		[\texttt{int\_literal}, tier=terminal]
		]
		[\texttt{less\_than}, tier=terminal]
		[$E$\textsuperscript{$type : \text{int}$}
		[\texttt{identifier}\textsuperscript{$name : \texttt{x}$}, tier=terminal]
		]
		]
	\end{forest}
\end{figure}

\newpage

Vegyünk egy másik példát, amiben egy \textit{újabb attribútumfajtá}ról lesz szó. A \textbf{ciklus} működését, valamint az abból való kiugrást (\texttt{break} utasítás) szemlélteti. Legyen a mondatunk
\[\texttt{while b do break done break},\]
aminek a nyelvtana
\begin{flalign*}
	& \prodrule{S}{L} \\
	& \prodrule{L}{\emptyword \mid UL} \\
	& \prodrule{U}{\texttt{break} \mid \texttt{while} ~ E ~ \texttt{do} ~ L ~ \texttt{done}} \\
	& \prodrule{E}{\texttt{identifier}}.
\end{flalign*}
A szintaxisfája nyilvánvalóan
\begin{figure}[h]
	\centering
	\begin{forest}
		for tree={edge={-}}
		[$S$
			[$L$
				[$U$
					[\texttt{while}, tier=terminal]
					[$E$
						[\texttt{identifier}, tier=terminal]
					]
					[\texttt{do}, tier=terminal]
					[$L$
						[$U$
							[\texttt{break}, tier=terminal]
						]
						[$L$
							[$\emptyword$, tier=terminal]
						]
					]
					[\texttt{done}, tier=terminal]
				]
				[$L$
					[$U$
						[\texttt{break}, tier=terminal]
					]
					[$L$
						[$\emptyword$, tier=terminal]
					]
				]
			]
		]
	\end{forest}
\end{figure}

Felvesszük a következő attribútumokat.
\begin{itemize}
	\item Az $L$ nemterminális szimbólumoknak (mint \textit{loop}) egy $in\_loop:\mathbb{B}$ attribútumot.
	\item Az $U$ szimbólumoknak (mint \textit{utasításlista}) szintén egy $in\_loop:\mathbb{B}$ attribútumot.
\end{itemize}

Mindkettő esetben a \textit{szabály jobb oldalán} állnak, amikor kiszámítjuk őket. Ugyanakkor fontos különbség, hogy \textbf{felülről lefelé} közvetít információt a szintaxisfában. Az ilyen attribútumokat \textbf{örökölt attribútum}oknak nevezzük, mivel a gyökerénél meghatározzuk ezt a tulajdonságot, ami az elemzés során ``leszivárog'' az alsóbb szintekre. Az egyes nyelvtani szabályainkhoz az alábbi \textbf{akciók}at rendeljük hozzá.

\begin{stuki*}[12cm]{$\prodrule{S}{L}$}
	\stm[2]{L.in\_loop := false \\
	\text{/* A legfelső szintű
		utasításlista nincs
		ciklus belsejében. */}}
\end{stuki*}

\begin{stuki*}[12cm]{$\prodrule{L}{\emptyword}$}
	\stm*[1]{/* Nincs teendő */}
\end{stuki*}

\begin{stuki*}[12cm]{$\prodrule{L_1}{UL_2}$}
	\stm{U.in\_loop := L_1.in\_loop}
	\stm{L_2.in\_loop := L_1.in\_loop}
	\stm*{/* Az utasításlista
		tulajdonsága
		öröklődik a lista
		elemeire. */}
\end{stuki*}

\begin{stuki*}[12cm]{$\prodrule{U}{\texttt{break}}$}
	\stm*{/* Ellenőrzés:
		A ‘break’ utasítás
		ciklus belsejében
		van? */}
	\begin{IF}{1}{\stm{\neg U.in\_loop}}
		\stm{error(\dots)}
		\ELSE
		\stm*{SKIP}
	\end{IF}
\end{stuki*}

\begin{stuki*}[12cm]{$\prodrule{U}{\texttt{while} ~ E ~ \texttt{do} ~ L ~ \texttt{done}}$}
	\stm[2]{L.in\_loop := true \\ 
	\text{/* A ciklus magja
		ciklus belsejében
		van. */}}
\end{stuki*}

\begin{stuki*}[12cm]{$\prodrule{E}{\texttt{identifier}}$}
	\stm*[1]{/* Más akciók */}
\end{stuki*}

Az attribútumokkal ellátott fa pedig:
\begin{figure}[h]
	\centering
	\begin{forest}
		for tree={edge={-}}
		[$S$
		[$L^{in\_loop : false}$
		[$U^{in\_loop : false}$
		[\texttt{while}, tier=terminal]
		[$E$
		[\texttt{identifier}, tier=terminal]
		]
		[\texttt{do}, tier=terminal]
		[$L^{in\_loop : true}$
		[$U^{in\_loop : true}$
		[\texttt{break} $\checkmark$, tier=terminal]
		]
		[$L^{in\_loop : true}$
		[$\emptyword$, tier=terminal]
		]
		]
		[\texttt{done}, tier=terminal]
		]
		[$L^{in\_loop : false}$
		[$U^{in\_loop : false}$
		[\texttt{break} $\lightning$, tier=terminal]
		]
		[$L^{in\_loop : false}$
		[$\emptyword$, tier=terminal]
		]
		]
		]
		]
	\end{forest}
\end{figure}

Ahogy láthatjuk, az első \texttt{break} utasítást helyesen használtuk, hiszen cikluson belül helyezkedik el, míg a második nem. Emiatt szemantikai hibát fog jelezni az elemző.

\newpage

A gyakorlatokon a Bison fordítógenerátort használjuk, aminek szemantikus elemzője kitüntetett szintetizált és szintetizált attribútumokat támogat, azonban örökölteket nem\footnote{Az ilyen nyelvtanokat $S$-attribútumnyelvtanoknak nevezzük.}. Emiatt jól illeszkedik az $LR$ elemzéshez, hiszen egy lépésben elvégzi a szintaktikus és szemantikus elemzést.

\begin{figure}
\begin{lstlisting}[style=cppstyle]
/*
  Nonterminal 'expression' has type 'type'.
*/
%type <type> expression

expression:
expression LESS_THAN expression
{
	/* 
	  $1, $2, $3, ... refer to 
	  the given attributes of the RHS of the rule. 
	*/
	if ($1 != int || $3 != int)
		error(...);
	else
		/* $$: the attribute of the LHS of the rule. */
		$$ = bool;
}
\end{lstlisting}
\caption{Példa a Bison szemantikus elemzőjére}
\end{figure}