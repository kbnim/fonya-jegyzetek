\chapter{A 2-es és 3-as nyelvcsalád viszonya}

A következő tételek szükséges feltételeket fogalmaznak meg a 3-as típusú
nyelvekre. Vannak nyelvek, amelyek bizonyíthatóan nem teljesítik a feltételeket,
de 2-es típusú grammatikával generálhatók.

\section{Szükséges feltétel 3-as típusú nyelvekre}

\begin{tcolorbox}
	\begin{lemma}[Kis Bar-Hillel lemma]
		$\forall L \in \mathcal{L}_3$ nyelvhez $\exists n \in \mathbb{N}^+$ \textbf{nyelvfüggő
		konstans}, hogy $\forall u \in L$, ahol $\ell(u) \geq n$ szó esetén van
		u-nak olyan $u = xyz$ felbontása, amelyre
		\begin{itemize}
			\item $\ell(xy) \leq n$,
			\item $y \neq \emptyword$,
			\item $\forall i \in \mathbb{N} : xy^iz \in L$.
		\end{itemize}
	\end{lemma}
\end{tcolorbox}

\textbf{\textit{Bizonyítás}}. // Kidolgozni.

\section{Szükséges és elégséges feltétel 3-as típusú nyelvekre}

\begin{tcolorbox}
	\begin{definition}[Maradéknyelv]
		Legyen $L$ egy $T$ ábácé felett értelmezett nyelv $(L \subseteq T^*)$.
		Az $L$ nyelv egy $p \in T^*$ \textbf{szóra értelmezett maradéknyelve} a következő:
		\[ L_p := \{ u \in T^* \mid pu \in L \} . \]
	\end{definition}
\end{tcolorbox}

\begin{tcolorbox}
	\begin{theorem}[Myhill--Nerode-tétel]
		Egy $L$ nyelv akkor és csak akkor 3-as típusú, ha a véges számú maradéknyelve van, azaz
		\[ L \in \mathcal{L}_3 ~ \Longleftrightarrow ~ \left| \left\{ L_p \setdivbar p \in T^* \right\} \right| < \infty . \]
	\end{theorem}
\end{tcolorbox}

\textbf{\textit{Megjegyzés}}. A szavakon egy osztályozást végzünk az
adott nyelvtől függően.

\textbf{\textit{Bizonyítás}}. // Kidolgozni.

%\section{Feladatok}